\documentclass[letterpaper]{article}
%\documentclass[letterpaper]{scrartcl}
\usepackage[margin=1.0in]{geometry}
\usepackage{graphicx}
\usepackage{mdwlist}
\usepackage{xcolor}
\usepackage{listings}
\usepackage{textcomp}

\lstset{basicstyle=\small\ttfamily,
  frame=single,
  backgroundcolor=\color{grey},
  showstringspaces=false,
  commentstyle=\color{blue},
  captionpos=b,
  keywordstyle=\color{blue}
}

\newcommand{\master}{master3}
\definecolor{grey}{rgb}{0.96,0.96,0.96}
%\captionsetup[lstlisting]{position=bottom}


\begin{document}

%\lfoot{\includegraphics[scale=0.5]{../figures/tron-approved}}
{\hfill\includegraphics[scale=0.14]{../figures/tron-approved}}

%\begin{titlepage}

\begin{center}
\vspace*{-0.5cm}
{\Large Cluster Building Recipes} \\ \vspace*{0.2cm}
{\large -- CentOS6.5 with FSP POC Repository -- } \\ \vspace*{0.75cm}

{\large \em Cluster Maker Team} \\
\today
\end{center}
%\end{titlepage}

%\maketitle

\section{Introduction}
This document endeavors to walk through a simple cluster installation using
components that are expected to be in FSP. This process is meant to be
repeatable on the Zeus cluster and requires access to {\em yum} repositories
that are currently hosted on the zeus head node. Note: this process is not
meant to dictate a specific FSP implementation direction; instead, it serves as a {\em
  proof-of-concept} highlighting interaction with an FSP repository and
coordination with a config management system. \\

\noindent {\bf Requirements}: this recipe assumes the availability of a head node
	  {\em master}, and two {\em compute} nodes. The {\em master} node is
	  provisioned with CentOS6.5 and is subsequently configured to
	  provision the remaining {\em compute} nodes with Warewulf in a
	  stateless configuration. We assume an external \$HOME file system is
	  available on master/computes via NFS and that compute node BMCs are
	  available via IPMI from the chosen master host. \\

\noindent For this recipe, we use the following nodes and network configuration
settings:

\begin{itemize*}
\item hostname=master2 , ip=192.168.0.5
\item hostname=c1, eth0(ip=192.168.0.40, mac=78:45:C4:F8:36:86), bmc=10.23.183.106
\item hostname=c2, eth0(ip=192.168.0.45, mac=78:45:C4:F8:36:41), bmc=10.23.183.105
\end{itemize*}

%%%\begin{itemize*}
%%%\item hostname=master3 , ip=192.168.0.10
%%%\item hostname=c1, eth0(ip=192.168.0.60, mac=78:45:C4:F8:35:42), bmc(ip=10.23.183.153)
%%%\item hostname=c2, eth0(ip=192.168.0.65, mac=78:45:C4:F8:36:3B), bmc(ip=10.23.183.118)
%%%\item hostname=c3, eth0(ip=192.168.0.70, mac=78:45:C4:F8:36:C8), bmc(ip=10.23.183.42)
%%%%\item hostname=c4, eth0(ip=192.168.0.75, mac=78:45:C4:F8:46:95), bmc(ip 
%%%\end{itemize*}



\section{Install Base Operating System(BOS)}

In an external setting, installing the BOS on a {\em master} host would
typically involve booting from a DVD iso image on a new server.  However, on
the Zeus cluster, several master nodes can be network booted with a minimal OS
install using {\em Warewulf}. The Warewulf configuration provisions the OS into
a ramdisk, configures local yum repositories to point to the FSP OBS server and
underlying distro repos, and mounts an NFS \$HOME file system. With this setup,
all that is required to achieve a clean BOS intall on a master host is a host
{\em reboot}. 

\section{Install FSP Components}

\subsection{Bootstrap}

To begin, bootstrap the master server with the bare essentials necessary to
enable a local configuration management system. The config management system
will then be used in subsequent commands to register additional FSP components
for installation.

% begin_fsp_run
% fsp_validation_comment Installing config management component

\begin{lstlisting}[language=bash,caption={Commands run on {\bf master node}.}]
[master]$ yum install -y losf
\end{lstlisting}

% end_fsp_run

\subsection{Add baseline FSP and components for {\em master} node}

As an example of the convenience that can be added via combination of config
management and a pre-provided FSP template, the next step adds all the minimum remaining
baseline FSP components necessary for a {\em master} server. We begin by
initializing the config mgmt system with an FSP-tuned template that defines
Warewulf as the underlying config management system and defined two node type classifications:
\begin{itemize*}
\item master
\item compute (e.g. c1, c2, c3, etc.)
\end{itemize*}
In this demo, we choose to store the cluster config files in /tmp. However, in
normal practice, this would likely be stored in a shared file system (with
config/template files stored in an SCM for versioning).

% begin_fsp_run
% fsp_validation_comment Installing FSP base components

\begin{lstlisting}[language=bash,keywords={}]
[master]$ . /etc/profile.d/losf.sh         # setup env
[master]$ export LOSF_CONFIG_DIR=/tmp/demo # path for configuration setup
[master]$ initconfig cluster FSP_base      # init config with FSP template
[master]$ losf addpkg -y lmod              # adding modules support
[master]$ losf addgroup -y FSP-warewulf    # adding Warewulf support
[master]$ update -q                        # install required components
\end{lstlisting}

% end_fsp_run

%%%Note that on a freshly installed system, the \texttt{update} process above will
%%%download and install all of the pre-defined FSP provided packages (and
%%%associated dependencies) registered in the config mgmt. template.
%%%
%%%\subsection{Complete basic Warewulf setup for {\em master} node}
%%%
%%%At this point, all of the packages necessary to use warewulf on the {\em master}
%%%host should be installed.  Next, we need to build the vNFS image, modify a few
%%%settings to work on the Zeus cluster, and provision a host. Example steps are
%%%as follows:
%%%
%%%\begin{lstlisting}[language=bash,literate={-}{-}1,keywords={},upquote=true]
%%%
%%%[master]$ unset BASH_ENV                  # Warewulf doesn't like BASH_ENV
%%%[master]$ wwbootstrap `uname -r`          # create warewulf bootstrap image
%%%[master]$ cluster-env                     # create warewulf ssh key
%%%
%%%[master]$ export CHROOT=/opt/fsp/admin/images/centos6.5/base
%%%
%%%# add new cluster key to base image
%%%
%%%[master]$ cat ~/.ssh/cluster.pub >> $CHROOT/root/.ssh/authorized_keys
%%%
%%%# add mount of $HOME and /opt/fsp/pub to base image
%%%
%%%[master]$ echo "192.168.0.1:/home /home nfs nfsvers=3 0 0" >> $CHROOT/etc/fstab
%%%[master]$ echo "192.168.0.5:/opt/fsp/pub /opt/fsp/pub nfs nfsvers=3 0 0" >> \
%%%    $CHROOT/etc/fstab
%%%
%%%# Export /opt/fsp/pub to cluster
%%%
%%%[master]$ echo "/opt/fsp/pub 192.168.0.40(ro)" >> /etc/exports
%%%[master]$ echo "/opt/fsp/pub 192.168.0.45(ro)" >> /etc/exports
%%%[master]$ exportfs -a
%%%
%%%# fix perm for munge
%%%
%%%[master]$ chown munge: $CHROOT/etc/munge/
%%%
%%%# build local vnfs (note: base chroot image provided by FSP)
%%%
%%%[master]$ wwvnfs  -y --chroot /opt/fsp/admin/images/centos6.5/base
%%%
%%%# Configure warewulf to allow provisioning over eth0
%%%
%%%[master]$ perl -pi -e "s/eth1/eth0/"  /etc/warewulf/provision.conf 
%%%\end{lstlisting}
%%%
%%%To enable SLURM, the local site administrators would need to create the
%%%\texttt{/etc/slurm/slurm.conf} file, customized to include desired compute
%%%nodes, queue settings, etc. A minimal example to setup SLURM from an example
%%%config file is below:
%%%
%%%\begin{lstlisting}[language=bash,keywords={}]
%%%
%%%# Create minimal SLURM config
%%%
%%%[master]$ head -n -2  /etc/slurm/slurm.conf.example  > /etc/slurm/slurm.conf
%%%[master]$ echo "PropagateResourceLimitsExcept=MEMLOCK" >> /etc/slurm/slurm.conf
%%%[master]$ echo "SlurmdLogFile=/var/log/slurm.log" >> /etc/slurm/slurm.conf
%%%[master]$ echo "NodeName=c[1-2] Sockets=2 CoresPerSocket=8 ThreadsPerCore=2 \
%%%     State=UNKNOWN" >> /etc/slurm/slurm.conf
%%%[master]$ echo "PartitionName=normal Nodes=c[1-2] Default=YES MaxTime=24:00:00 \
%%%     State=UP" >> /etc/slurm/slurm.conf
%%%
%%%[master]$ perl -pi -e "s/ControlMachine=\S+/ControlMachine=master/" 
%%%     /etc/slurm/slurm.conf
%%%\end{lstlisting}
%%%
%%%In preparation for provisioning compute nodes, we can now register the desired
%%%network settings for two example compute nodes. Recall that this example is
%%%using the compute nodes assigned to master. If using a different master host,
%%%adjust the IP and MAC addresses accordingly.
%%%
%%%\begin{lstlisting}[language=bash,keywords={},upquote=true]
%%%# Register new nodes: c1, c2
%%%
%%%[master]$ wwsh -y node new c1 --netdev=eth0 --ipaddr=192.168.0.40 
%%%     --hwaddr=78:45:C4:F8:36:86
%%%[master]$ wwsh -y node new c2 --netdev=eth0 --ipaddr=192.168.0.45
%%%     --hwaddr=78:45:C4:F8:36:41
%%%
%%%# Define vNFS image
%%%
%%%[master]$ wwsh -y provision set c[1-2] --vnfs=base --bootstrap=`uname -r`
%%%
%%%# Update DHCP for Warewulf
%%%
%%%[master]$ wwsh dhcp update	
%%%[master]$ wwsh dhcp restart
%%%
%%%# Import user credentials
%%%
%%%[master]$ wwsh file import /etc/passwd
%%%[master]$ wwsh file import /etc/group
%%%[master]$ wwsh file import /etc/shadow
%%%
%%%# Import SLURM config and MUNGE key
%%%
%%%[master]$ wwsh file import /etc/slurm/slurm.conf
%%%[master]$ wwsh file import /etc/munge/munge.key 
%%%
%%%# Export credentials, hosts, and slurm-related files to computes
%%%
%%%[master]$ wwsh -y provision set c[1-2] 
%%%     --files=dynamic_hosts,passwd,group,shadow,slurm.conf,munge.key
%%%\end{lstlisting}
%%%
%%%
%%%\subsection{Boot compute nodes}
%%%
%%%At this point, the {\em master} server should be able to boot the newly defined
%%%compute nodes.  The service processors for the compute hosts are available via a separate
%%%network on the Zeus cluster. You can point a web browser to their respective
%%%IPs to reboot, or you can issue ipmi commands from the Zeus head node.  An example to reboot hosts {\em c1}
%%%and {\em c2} from the Zeus {\em headnode} is shown below. Note that mnode09 and
%%%mnode10 correspond to c1 and c2 for this example (see /etc/motd on Zeus
%%%headnode for the node mapping).
%%%
%%%%%%On master, you can access the out-of-band BMC network on the
%%%%%%second interface (eth1). Once enabled, you can use ipmi to restart the compute
%%%%%%nodes to boot from this {\em master} host.
%%%\vspace*{0.25cm}
%%%%%%\begin{lstlisting}[language=bash]
%%%%%%[master]$ ifup eth1 
%%%%%%[master]$ ireset 10.23.183.153      # power cycles c1 through BMC IP
%%%%%%[master]$ ireset 10.23.183.118      # power cycles c2 through BMC IP
%%%%%%\end{lstlisting} 
%%%
%%%\begin{lstlisting}[language=bash]
%%%[zeusheadnode]$ ireset mnode09
%%%[zeusheadnode]$ ireset mnode10
%%%\end{lstlisting}
%%%
%%%\vspace*{0.25cm}
%%%Once kicked off, the boot process should take less than 5 minues and you can
%%%verify that the compute hosts are available via ssh, or via parallel ssh tools to multiple
%%%hosts. For example:
%%%
%%%\vspace*{0.25cm}
%%%\begin{lstlisting}[language=bash]
%%%[master]$ koomie_cf -x c[1-2] uptime
%%%c1  14:59:46 up 1 min,  0 users,  load average: 0.12, 0.05, 0.02
%%%c2  14:59:32 up 1 min,  0 users,  load average: 0.23, 0.08, 0.03
%%%\end{lstlisting}
%%%
%%%%%%\section{Local Site Config}
%%%
%%%%%%The above sections highlighted the basic steps required to be able to provision
%%%%%%an FSP provided image on compute nodes. The next steps for local site
%%%%%%customization would include:
%%%%%%
%%%%%%\begin{itemize*}
%%%%%%\item Update /etc/fstab for vNFS image to mount \$HOME
%%%%%%\item Create desired SLURM settings in /etc/slurm.conf and distribute
%%%%%%\end{itemize*}
%%%%%%
%%%%\section{Add compilers/MPI}
%%%
%%%%\section{Configure SLURM}
%%%
%%%%\section{Configure and Boot Compute Nodes}
%%%
%%%\section{Install Licenses for Intel Software}
%%%
%%%For internal testing convenience, an RPM containing valid licenses for Intel
%%%compilers and MPI is available. To install, issue:
%%%
%%%\vspace*{0.25cm}
%%%\begin{lstlisting}[language=bash,keywords={},upquote=true]
%%%[master]$ losf addpkg FSP-licenses
%%%[master]$ update -q
%%%\end{lstlisting}
%%%
%%%\section{Run a Test Job}
%%%
%%%At this point, the cluster should be available to run jobs. 
%%%
%%%\begin{lstlisting}[language=bash]
%%%
%%%# Start SLURM on master server
%%%
%%%[master]$ service slurm start
%%%
%%%# Open hosts for production
%%%
%%%[master]$ scontrol  update state=idle nodename=c[1-2]
%%%
%%%# Run a test job as a user
%%%
%%%[master]$ su - kwschulz
%%%[kwschulz@master ~]$ mpicc hello.c
%%%
%%%[kwschulz@master ~]$ srun -n 8 -N 2 ./a.out 
%%%
%%% Hello, world (8 procs total)
%%%    --> Process #   0 of   8 is alive. ->c1
%%%    --> Process #   1 of   8 is alive. ->c1
%%%    --> Process #   2 of   8 is alive. ->c1
%%%    --> Process #   3 of   8 is alive. ->c1
%%%    --> Process #   4 of   8 is alive. ->c2
%%%    --> Process #   5 of   8 is alive. ->c2
%%%    --> Process #   6 of   8 is alive. ->c2
%%%    --> Process #   7 of   8 is alive. ->c2
%%%\end{lstlisting}
%%%
\end{document}

