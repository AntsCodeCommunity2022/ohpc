\begin{abstract}
OpenHPC is a newly formed, community-based project
%Linux Foundation collaborative Project whose
that is providing an integrated collection of HPC-centric software
components that can be used to implement a full-featured reference HPC compute
resource. Components span the entire HPC software ecosystem including
provisioning and system administration tools, resource management, I/O
services, development tools, numerical libraries, and performance analysis
tools.

Common clustering tools and scientific libraries are distributed as pre-built
and validated binaries and are meant to seamlessly layer on top of existing
Linux distributions. The architecture of OpenHPC is intentionally modular to
allow end users to pick and choose from the provided components, as well as to
foster a community of open contribution. This paper presents an overview of the
underlying community vision, governance structure, packaging conventions, build
and release infrastructure and validation methodologies.

\end{abstract}
