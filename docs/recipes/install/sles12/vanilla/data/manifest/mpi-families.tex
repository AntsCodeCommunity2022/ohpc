\newcolumntype{C}[1]{>{\centering}p{#1}} 
\newcolumntype{L}[1]{>{\raggedleft}p{#1}} 
\small
\begin{tabularx}{\textwidth}{L{\firstColWidth{}}|C{\secondColWidth{}}|X}
\toprule
{\bf RPM Package Name} & {\bf Version} & {\bf Info/URL}  \\ 
\midrule

% <-- begin entry for intel-mpi-devel-fsp
\multirow{2}{*}{intel-mpi-devel-fsp} & 
\multirow{2}{*}{5.1.0.069} & 
Intel(R) MPI Library for Linux* OS. \newline { \color{blue} https://software.intel.com/en-us/intel-mpi-library} 
\\ \hline 
% <-- end entry for %name_base

% <-- begin entry for intel-mpi-fsp
\multirow{2}{*}{intel-mpi-fsp} & 
\multirow{2}{*}{5.1.0.069} & 
Intel(R) MPI Library for Linux* OS. \newline { \color{blue} https://software.intel.com/en-us/intel-mpi-library} 
\\ \hline 
% <-- end entry for %name_base

% <-- begin entry for mvapich2
mvapich2-gnu-fsp & 
\multirow{2}{*}{2.1} & 
\multirow{2}{\linewidth}{OSU MVAPICH2 MPI implementation. \newline {\color{blue} http://mvapich.cse.ohio-state.edu/overview/mvapich2}} \\ 
mvapich2-intel-fsp & 
& \\ 
\hline
% <-- end entry for mvapich2

% <-- begin entry for openmpi
openmpi-gnu-fsp & 
\multirow{2}{*}{1.8.4} & 
\multirow{2}{\linewidth}{A powerful implementation of MPI. \newline {\color{blue} http://www.open-mpi.org}} \\ 
openmpi-intel-fsp & 
& \\ 
\hline
% <-- end entry for openmpi

\bottomrule
\end{tabularx}
