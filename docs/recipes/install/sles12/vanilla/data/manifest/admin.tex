\newcolumntype{C}[1]{>{\centering}p{#1}} 
\newcolumntype{L}[1]{>{\raggedleft}p{#1}} 
\small
\begin{tabularx}{\textwidth}{L{\firstColWidth{}}|C{\secondColWidth{}}|X}
\toprule
{\bf RPM Package Name} & {\bf Version} & {\bf Info/URL}  \\ 
\midrule

% <-- begin entry for conman-fsp
\multirow{2}{*}{conman-fsp} & 
\multirow{2}{*}{0.2.7} & 
ConMan: The Console Manager. \newline { \color{blue} http://conman.googlecode.com} 
\\ \hline 
% <-- end entry for %name_base

% <-- begin entry for docs-fsp
\multirow{2}{*}{docs-fsp} & 
\multirow{2}{*}{15.31} & 
\multirow{2}{*}{Forest Peak documentation.} \\
& & 
\\ \hline 
% <-- end entry for %name_base

% <-- begin entry for examples-fsp
\multirow{2}{*}{examples-fsp} & 
\multirow{2}{*}{1.2} & 
\multirow{2}{*}{Example source code and templates for use within FSP environment.} \\
& & 
\\ \hline 
% <-- end entry for %name_base

% <-- begin entry for intel-clck-fsp
\multirow{2}{*}{intel-clck-fsp} & 
\multirow{2}{*}{2.2.2} & 
Intel(R) Cluster Checker. \newline { \color{blue} http://intel.com/go/cluster} 
\\ \hline 
% <-- end entry for %name_base

% <-- begin entry for lmod-defaults
lmod-defaults-intel-fsp & 
\multirow{1}{*}{1.1} & 
\multirow{1}{\linewidth}{FSP default login environment.} \\ 
\hline
% <-- end entry for lmod-defaults

% <-- begin entry for lmod-fsp
\multirow{2}{*}{lmod-fsp} & 
\multirow{2}{*}{5.9.3} & 
Lua based Modules (lmod). \newline { \color{blue} https://github.com/TACC/Lmod} 
\\ \hline 
% <-- end entry for %name_base

% <-- begin entry for losf-fsp
\multirow{2}{*}{losf-fsp} & 
\multirow{2}{*}{0.52.0} & 
A Linux operating system framework for managing HPC clusters.  { \color{blue} https://github.com/hpcsi/losf} 
\\ \hline 
% <-- end entry for %name_base

% <-- begin entry for pdsh-fsp
\multirow{2}{*}{pdsh-fsp} & 
\multirow{2}{*}{2.31} & 
Parallel remote shell program. \newline { \color{blue} http://sourceforge.net/projects/pdsh} 
\\ \hline 
% <-- end entry for %name_base

% <-- begin entry for prun-fsp
\multirow{2}{*}{prun-fsp} & 
\multirow{2}{*}{0.1.0} & 
\multirow{2}{*}{Convenience utility for parallel job launch.} \\
& & 
\\ \hline 
% <-- end entry for %name_base

\bottomrule
\end{tabularx}
