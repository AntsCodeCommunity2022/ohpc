\clearpage

\definecolor{Gray}{gray}{0.5}
\newcommand{\captionSpace}{-0.15cm}
\newcommand{\tabSpaceBot}{1.0cm}
\captionsetup{justification=raggedright,singlelinecheck=false}

\appendix
\section*{Appendix - Package Manifest} \label{appendix:manifest}
\addcontentsline{toc}{section}{Appendix - Package Manifest}
\renewcommand{\thesubsection}{\Alph{subsection}}

\vspace*{0.25cm}
This appendix provides a summary of available convenience groups and all of the
underlying RPM packages that are available as part of this \FSP{} release. The
convenience groups are aliases that the underlying package manager supports in
order to provide a mechanism to group related collections of packages
together. The collection of packages (along with any additional dependencies)
can be installed by using the group name. A list of the available convenience
groups and a brief description are presented in Table~\ref{table:groups}.

\vspace*{1.25cm}
\begin{table}[h] 
\caption{\bf Available \FSP{} Convenience Groups} \vspace*{\captionSpace{}}
\label{table:groups}
\input data/manifest/patterns
\end{table}

\newpage
What follows next in this Appendix are a series of tables that summarize the
underlying RPM packages available in this \FSP{} release. These packages are
organized by groupings based on their general functionality and each table
provides information for the specific RPM name, version, brief summary, and the
web URL where additional information can be contained for the component. Note
that many of the 3rd party community libraries that are pre-packaged
with \FSP{} are built using multiple compiler and MPI families. In these cases,
the RPM package name includes delimiters identifying the development
environment for which each package build is targeted.  Additional information
on the \FSP{} package naming scheme is presented in \S\ref{sec:3rdparty}. 
The relevant package groupings and associated Table reference are as follows:

\vspace*{0.1cm}

\begin{itemize*}
\item Administrative tools (Table~\ref{table:admin})
\item Provisioning (Table~\ref{table:provisioning})
\item Resource management (Table~\ref{table:rms})
\item Compiler families (Table~\ref{table:compiler-families})
\item MPI families (Table~\ref{table:mpi-families})
\item Development tools (Table~\ref{table:dev-tools})
\item Performance analysis tools (Table~\ref{table:perf-tools})
\item Distro support packages and dependencies (Table~\ref{table:distro-packages})
\item IO Libraries (Table~\ref{table:io-libs})
\item Serial Libraries (Table~\ref{table:serial-libs})
\item Parallel Libraries (Table~\ref{table:parallel-libs})
\end{itemize*}

\newcommand{\firstColWidth}{3.5cm}
\newcommand{\secondColWidth}{1.5cm}

\vspace*{1.0cm}

% Administration Tools 
\begin{table}[h]
\caption{\bf Administrative Tools} \vspace*{\captionSpace{}} \label{table:admin}
\input data/manifest/admin
\end{table}
\vspace*{0.5cm}

\renewcommand{\firstColWidth}{4.5cm}
\renewcommand{\secondColWidth}{2.0cm}

% Provisioning

\begin{table}[h!]
\caption{\bf Provisioning} \vspace*{\captionSpace{}} \label{table:provisioning}
\input data/manifest/provisioning
\vspace*{\tabSpaceBot{}}
\end{table} 

% Resource Management
\begin{table}[h!]
\caption{\bf Resource Management} \vspace*{\captionSpace{}} \label{table:rms}
\input data/manifest/rms
\vspace*{\tabSpaceBot{}}
\end{table}

% Compiler Families
\begin{table}[h!]
\caption{\bf Compiler Families} \vspace*{\captionSpace{}} \label{table:compiler-families}
\input data/manifest/compiler-families
\vspace*{\tabSpaceBot{}}
\end{table}

% MPI Families
\begin{table}[h!]
\caption{\bf MPI Families} \vspace*{\captionSpace{}} \label{table:mpi-families}
\input data/manifest/mpi-families
\vspace*{\tabSpaceBot{}}
\end{table}

\renewcommand{\firstColWidth}{4.5cm}

% Development Tools
\begin{table}[h!]
\caption{\bf Development Tools} \vspace*{\captionSpace{}} \label{table:dev-tools}
\input data/manifest/dev-tools
\vspace*{\tabSpaceBot{}}
\end{table}

% Perf Tools
\begin{table}[h!]
\caption{\bf Performance Analysis Tools} \vspace*{\captionSpace{}} \label{table:perf-tools}
\input data/manifest/perf-tools
\vspace*{\tabSpaceBot{}}
\end{table}

% Distro Packages
\begin{table}[h!]
\caption{\bf Distro Support Packages/Dependencies} \vspace*{\captionSpace{}} \label{table:distro-packages}
\input data/manifest/distro-packages
\vspace*{\tabSpaceBot{}}
\end{table}

\renewcommand{\firstColWidth}{4.75cm}
\renewcommand{\secondColWidth}{1.75cm}

% Lustre
\begin{table}[h!]
\caption{\bf Lustre} \vspace*{\captionSpace{}} \label{table:lustre}
\input data/manifest/lustre
\vspace*{\tabSpaceBot{}}
\end{table}

% IO Libs
\begin{table}[h!]
\caption{\bf IO Libraries} \vspace*{\captionSpace{}} \label{table:io-libs}
\input data/manifest/io-libs
\vspace*{\tabSpaceBot{}}
\end{table}

% Serial libs
\begin{table}[h!]
\caption{\bf Serial Libraries} \vspace*{\captionSpace{}} \label{table:serial-libs}
\input data/manifest/serial-libs
\vspace*{\tabSpaceBot{}}
\end{table}

% Parallel libs
\begin{table}[h!]
\caption{\bf Parallel Libraries} \vspace*{\captionSpace{}} \label{table:parallel-libs}
\input data/manifest/parallel-libs
\vspace*{\tabSpaceBot{}}
\end{table}





