\newcolumntype{C}[1]{>{\centering}p{#1}} 
\newcolumntype{L}[1]{>{\raggedleft}p{#1}} 
\small
\begin{tabularx}{\textwidth}{L{\firstColWidth{}}|C{\secondColWidth{}}|X}
\toprule
{\bf RPM Package Name} & {\bf Version} & {\bf Info/URL}  \\ 
\midrule

mpiP-gnu-impi-ohpc &
\multirow{18}{*}{3.4.1} & 
\multirow{18}{\linewidth}{mpiP: a lightweight profiling library for MPI applications. \newline {\color{logoblue} \url{http://mpip.sourceforge.net}}} \\ 
mpiP-gnu-mpich-ohpc &
& \\ 
mpiP-gnu-mvapich2-ohpc &
& \\ 
mpiP-gnu-openmpi-ohpc &
& \\ 
mpiP-gnu7-impi-ohpc &
& \\ 
mpiP-gnu7-mpich-ohpc &
& \\ 
mpiP-gnu7-mvapich2-ohpc &
& \\ 
mpiP-gnu7-openmpi-ohpc &
& \\ 
mpiP-gnu7-openmpi3-ohpc &
& \\ 
mpiP-gnu8-impi-ohpc &
& \\ 
mpiP-gnu8-mpich-ohpc &
& \\ 
mpiP-gnu8-mvapich2-ohpc &
& \\ 
mpiP-gnu8-openmpi3-ohpc &
& \\ 
mpiP-intel-impi-ohpc &
& \\ 
mpiP-intel-mpich-ohpc &
& \\ 
mpiP-intel-mvapich2-ohpc &
& \\ 
mpiP-intel-openmpi-ohpc &
& \\ 
mpiP-intel-openmpi3-ohpc &
& \\ 
\hline
% <-- end entry for mpiP

% <-- begin entry for msr-safe-ohpc
\multirow{2}{*}{msr-safe-ohpc} & 
\multirow{2}{*}{1.2.0} & 
Allows safer access to model specific registers (MSRs). \newline { \color{logoblue} \url{https://github.com/LLNL/msr-safe}} 
\\ \hline 
% <-- end entry for msr-safe-ohpc

% <-- begin entry for msr-safe-ohpc-kmp-default
\multirow{2}{*}{msr-safe-ohpc-kmp-default} & 
\multirow{2}{*}{1.2.0\_k4.4.73\_5} & 
Allows safer access to model specific registers (MSRs). \newline { \color{logoblue} \url{https://github.com/LLNL/msr-safe}} 
\\ \hline 
% <-- end entry for msr-safe-ohpc-kmp-default

% <-- begin entry for paraver-ohpc
\multirow{2}{*}{paraver-ohpc} & 
\multirow{2}{*}{4.7.2} & 
Paraver. \newline { \color{logoblue} \url{https://tools.bsc.es}} 
\\ \hline 
% <-- end entry for paraver-ohpc

% <-- begin entry for papi-ohpc
\multirow{2}{*}{papi-ohpc} & 
\multirow{2}{*}{5.6.0} & 
Performance Application Programming Interface. \newline { \color{logoblue} \url{http://icl.cs.utk.edu/papi}} 
\\ \hline 
% <-- end entry for papi-ohpc

% <-- begin entry for pdtoolkit
pdtoolkit-gnu-ohpc &
\multirow{4}{*}{3.25} & 
\multirow{4}{\linewidth}{PDT is a framework for analyzing source code. \newline {\color{logoblue} \url{http://www.cs.uoregon.edu/Research/pdt}}} \\ 
pdtoolkit-gnu7-ohpc &
& \\ 
pdtoolkit-gnu8-ohpc &
& \\ 
pdtoolkit-intel-ohpc &
& \\ 
\hline
% <-- end entry for pdtoolkit

% <-- begin entry for scalasca
scalasca-gnu-impi-ohpc &
\multirow{10}{*}{2.3.1} & 
\multirow{18}{\linewidth}{Toolset for performance analysis of large-scale parallel applications. \newline {\color{logoblue} \url{http://www.scalasca.org}}} \\ 
scalasca-gnu-mpich-ohpc &
& \\ 
scalasca-gnu-mvapich2-ohpc &
& \\ 
scalasca-gnu-openmpi-ohpc &
& \\ 
scalasca-gnu7-impi-ohpc &
& \\ 
scalasca-gnu7-mpich-ohpc &
& \\ 
scalasca-gnu7-mvapich2-ohpc &
& \\ 
scalasca-gnu7-openmpi-ohpc &
& \\ 
scalasca-gnu7-openmpi3-ohpc &
& \\ 
scalasca-intel-openmpi-ohpc &
& \\ 
\cline{1-2} scalasca-gnu8-impi-ohpc &\multirow{8}{*}{2.4}
& \\ 
scalasca-gnu8-mpich-ohpc &
& \\ 
scalasca-gnu8-mvapich2-ohpc &
& \\ 
scalasca-gnu8-openmpi3-ohpc &
& \\ 
scalasca-intel-impi-ohpc &
& \\ 
scalasca-intel-mpich-ohpc &
& \\ 
scalasca-intel-mvapich2-ohpc &
& \\ 
scalasca-intel-openmpi3-ohpc &
& \\ 
\hline
% <-- end entry for scalasca

% <-- begin entry for scorep
\bottomrule
\end{tabularx}
