\newcolumntype{C}[1]{>{\centering}p{#1}} 
\newcolumntype{L}[1]{>{\raggedleft}p{#1}} 
\small
\begin{tabularx}{\textwidth}{L{\firstColWidth{}}|C{\secondColWidth{}}|X}
\toprule
{\bf RPM Package Name} & {\bf Version} & {\bf Info/URL}  \\ 
\midrule

netcdf-fortran-gnu-impi-ohpc &
\multirow{18}{*}{4.4.4} & 
\multirow{18}{\linewidth}{Fortran Libraries for the Unidata network Common Data Form. \newline {\color{logoblue} \url{http://www.unidata.ucar.edu/software/netcdf}}} \\ 
netcdf-fortran-gnu-mpich-ohpc &
& \\ 
netcdf-fortran-gnu-mvapich2-ohpc &
& \\ 
netcdf-fortran-gnu-openmpi-ohpc &
& \\ 
netcdf-fortran-gnu7-impi-ohpc &
& \\ 
netcdf-fortran-gnu7-mpich-ohpc &
& \\ 
netcdf-fortran-gnu7-mvapich2-ohpc &
& \\ 
netcdf-fortran-gnu7-openmpi-ohpc &
& \\ 
netcdf-fortran-gnu7-openmpi3-ohpc &
& \\ 
netcdf-fortran-gnu8-impi-ohpc &
& \\ 
netcdf-fortran-gnu8-mpich-ohpc &
& \\ 
netcdf-fortran-gnu8-mvapich2-ohpc &
& \\ 
netcdf-fortran-gnu8-openmpi3-ohpc &
& \\ 
netcdf-fortran-intel-impi-ohpc &
& \\ 
netcdf-fortran-intel-mpich-ohpc &
& \\ 
netcdf-fortran-intel-mvapich2-ohpc &
& \\ 
netcdf-fortran-intel-openmpi-ohpc &
& \\ 
netcdf-fortran-intel-openmpi3-ohpc &
& \\ 
\hline
% <-- end entry for netcdf-fortran

% <-- begin entry for netcdf
netcdf-gnu-impi-ohpc &
\multirow{6}{*}{4.5.0} & 
\multirow{18}{\linewidth}{C Libraries for the Unidata network Common Data Form. \newline {\color{logoblue} \url{http://www.unidata.ucar.edu/software/netcdf}}} \\ 
netcdf-gnu-mpich-ohpc &
& \\ 
netcdf-gnu-mvapich2-ohpc &
& \\ 
netcdf-gnu-openmpi-ohpc &
& \\ 
netcdf-gnu7-openmpi-ohpc &
& \\ 
netcdf-intel-openmpi-ohpc &
& \\ 
\cline{1-2} netcdf-gnu7-impi-ohpc &\multirow{12}{*}{4.6.1}
& \\ 
netcdf-gnu7-mpich-ohpc &
& \\ 
netcdf-gnu7-mvapich2-ohpc &
& \\ 
netcdf-gnu7-openmpi3-ohpc &
& \\ 
netcdf-gnu8-impi-ohpc &
& \\ 
netcdf-gnu8-mpich-ohpc &
& \\ 
netcdf-gnu8-mvapich2-ohpc &
& \\ 
netcdf-gnu8-openmpi3-ohpc &
& \\ 
netcdf-intel-impi-ohpc &
& \\ 
netcdf-intel-mpich-ohpc &
& \\ 
netcdf-intel-mvapich2-ohpc &
& \\ 
netcdf-intel-openmpi3-ohpc &
& \\ 
\hline
% <-- end entry for netcdf

% <-- begin entry for phdf5
\bottomrule
\end{tabularx}
