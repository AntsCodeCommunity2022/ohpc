\newcolumntype{C}[1]{>{\centering}p{#1}} 
\newcolumntype{L}[1]{>{\raggedleft}p{#1}} 
\newcolumntype{C}[1]{>{\centering}p{#1}} 
\newcolumntype{L}[1]{>{\raggedleft}p{#1}} 
\small
\begin{tabularx}{\textwidth}{L{\firstColWidth{}}|C{\secondColWidth{}}|X}
\toprule
{\bf RPM Package Name} & {\bf Version} & {\bf Info/URL}  \\ 
\midrule

% <-- begin entry for dimemas
dimemas-gnu9-mpich-ohpc &
\multirow{2}{*}{5.4.1} & 
\multirow{2}{\linewidth}{Dimemas tool. \newline {\color{logoblue} \url{https://tools.bsc.es}}} \\ 
dimemas-gnu9-openmpi4-ohpc &
& \\ 
\hline
% <-- end entry for dimemas

% <-- begin entry for extrae
extrae-gnu9-mpich-ohpc &
\multirow{2}{*}{3.7.0} & 
\multirow{2}{\linewidth}{Extrae tool. \newline {\color{logoblue} \url{https://tools.bsc.es}}} \\ 
extrae-gnu9-openmpi4-ohpc &
& \\ 
\hline
% <-- end entry for extrae

% <-- begin entry for imb
imb-arm1-mpich-ohpc &
\multirow{1}{*}{2018.1} & 
\multirow{3}{\linewidth}{Intel MPI Benchmarks (IMB). \newline {\color{logoblue} \url{https://software.intel.com/en-us/articles/intel-mpi-benchmarks}}} \\ 
\cline{1-2} imb-gnu9-mpich-ohpc &\multirow{2}{*}{2019.6}
& \\ 
imb-gnu9-openmpi4-ohpc &
& \\ 
\hline
% <-- end entry for imb

% <-- begin entry for omb
omb-arm1-mpich-ohpc &
\multirow{4}{*}{5.6.2} & 
\multirow{4}{\linewidth}{OSU Micro-benchmarks. \newline {\color{logoblue} \url{http://mvapich.cse.ohio-state.edu/benchmarks}}} \\ 
omb-arm1-openmpi4-ohpc &
& \\ 
omb-gnu9-mpich-ohpc &
& \\ 
omb-gnu9-openmpi4-ohpc &
& \\ 
\hline
% <-- end entry for omb

% <-- begin entry for paraver-ohpc
\multirow{2}{*}{paraver-ohpc} & 
\multirow{2}{*}{4.8.2} & 
Paraver. \newline { \color{logoblue} \url{https://tools.bsc.es}} 
\\ \hline 
% <-- end entry for paraver-ohpc

% <-- begin entry for papi-ohpc
\multirow{2}{*}{papi-ohpc} & 
\multirow{2}{*}{5.7.0} & 
Performance Application Programming Interface. \newline { \color{logoblue} \url{http://icl.cs.utk.edu/papi}} 
\\ \hline 
% <-- end entry for papi-ohpc

% <-- begin entry for pdtoolkit
pdtoolkit-arm1-ohpc &
\multirow{2}{*}{3.25.1} & 
\multirow{2}{\linewidth}{PDT is a framework for analyzing source code. \newline {\color{logoblue} \url{http://www.cs.uoregon.edu/Research/pdt}}} \\ 
pdtoolkit-gnu9-ohpc &
& \\ 
\hline
% <-- end entry for pdtoolkit

% <-- begin entry for scalasca
scalasca-gnu9-mpich-ohpc &
\multirow{2}{*}{2.5} & 
\multirow{2}{\linewidth}{Toolset for performance analysis of large-scale parallel applications.  {\color{logoblue} \url{http://www.scalasca.org}}} \\ 
scalasca-gnu9-openmpi4-ohpc &
& \\ 
\hline
% <-- end entry for scalasca

% <-- begin entry for scorep
scorep-gnu9-mpich-ohpc &
\multirow{2}{*}{6.0} & 
\multirow{2}{\linewidth}{Scalable Performance Measurement Infrastructure for Parallel Codes.  {\color{logoblue} \url{http://www.vi-hps.org/projects/score-p}}} \\ 
scorep-gnu9-openmpi4-ohpc &
& \\ 
\hline
% <-- end entry for scorep

% <-- begin entry for tau
tau-gnu9-mpich-ohpc &
\multirow{2}{*}{2.29} & 
\multirow{2}{\linewidth}{Tuning and Analysis Utilities Profiling Package. \newline {\color{logoblue} \url{http://www.cs.uoregon.edu/research/tau/home.php}}} \\ 
tau-gnu9-openmpi4-ohpc &
& \\ 
\hline
% <-- end entry for tau

\bottomrule
\end{tabularx}
