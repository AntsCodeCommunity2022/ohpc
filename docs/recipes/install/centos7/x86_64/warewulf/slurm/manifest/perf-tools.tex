\newcolumntype{C}[1]{>{\centering}p{#1}} 
\newcolumntype{L}[1]{>{\raggedleft}p{#1}} 
\newcolumntype{C}[1]{>{\centering}p{#1}} 
\newcolumntype{L}[1]{>{\raggedleft}p{#1}} 
\small
\begin{tabularx}{\textwidth}{L{\firstColWidth{}}|C{\secondColWidth{}}|X}
\toprule
{\bf RPM Package Name} & {\bf Version} & {\bf Info/URL}  \\ 
\midrule

% <-- begin entry for imb
imb-gnu7-openmpi-ohpc &
\multirow{1}{*}{4.1} & 
\multirow{14}{\linewidth}{Intel MPI Benchmarks (IMB). \newline {\color{logoblue} \url{https://software.intel.com/en-us/articles/intel-mpi-benchmarks}}} \\ 
\cline{1-2} imb-gnu-impi-ohpc &\multirow{13}{*}{2018.1}
& \\ 
imb-gnu-mpich-ohpc &
& \\ 
imb-gnu-mvapich2-ohpc &
& \\ 
imb-gnu-openmpi-ohpc &
& \\ 
imb-gnu7-impi-ohpc &
& \\ 
imb-gnu7-mpich-ohpc &
& \\ 
imb-gnu7-mvapich2-ohpc &
& \\ 
imb-gnu7-openmpi3-ohpc &
& \\ 
imb-intel-impi-ohpc &
& \\ 
imb-intel-mpich-ohpc &
& \\ 
imb-intel-mvapich2-ohpc &
& \\ 
imb-intel-openmpi-ohpc &
& \\ 
imb-intel-openmpi3-ohpc &
& \\ 
\hline
% <-- end entry for imb

% <-- begin entry for likwid
likwid-gnu7-ohpc &
\multirow{2}{*}{4.3.2} & 
\multirow{2}{\linewidth}{Toolsuite of command line applications for performance oriented programmers.  {\color{logoblue} \url{https://github.com/RRZE-HPC/likwid}}} \\ 
likwid-intel-ohpc &
& \\ 
\hline
% <-- end entry for likwid

% <-- begin entry for mpiP
mpiP-gnu-impi-ohpc &
\multirow{14}{*}{3.4.1} & 
\multirow{14}{\linewidth}{mpiP: a lightweight profiling library for MPI applications. \newline {\color{logoblue} \url{http://mpip.sourceforge.net}}} \\ 
mpiP-gnu-mpich-ohpc &
& \\ 
mpiP-gnu-mvapich2-ohpc &
& \\ 
mpiP-gnu-openmpi-ohpc &
& \\ 
mpiP-gnu7-impi-ohpc &
& \\ 
mpiP-gnu7-mpich-ohpc &
& \\ 
mpiP-gnu7-mvapich2-ohpc &
& \\ 
mpiP-gnu7-openmpi-ohpc &
& \\ 
mpiP-gnu7-openmpi3-ohpc &
& \\ 
mpiP-intel-impi-ohpc &
& \\ 
mpiP-intel-mpich-ohpc &
& \\ 
mpiP-intel-mvapich2-ohpc &
& \\ 
mpiP-intel-openmpi-ohpc &
& \\ 
mpiP-intel-openmpi3-ohpc &
& \\ 
\hline
% <-- end entry for mpiP

% <-- begin entry for papi-ohpc
\multirow{2}{*}{papi-ohpc} & 
\multirow{2}{*}{5.6.0} & 
Performance Application Programming Interface. \newline { \color{logoblue} \url{http://icl.cs.utk.edu/papi}} 
\\ \hline 
% <-- end entry for papi-ohpc

% <-- begin entry for pdtoolkit
pdtoolkit-gnu-ohpc &
\multirow{3}{*}{3.25} & 
\multirow{3}{\linewidth}{PDT is a framework for analyzing source code. \newline {\color{logoblue} \url{http://www.cs.uoregon.edu/Research/pdt}}} \\ 
pdtoolkit-gnu7-ohpc &
& \\ 
pdtoolkit-intel-ohpc &
& \\ 
\hline
% <-- end entry for pdtoolkit

% <-- begin entry for scalasca
scalasca-gnu-impi-ohpc &
\multirow{14}{*}{2.3.1} & 
\multirow{14}{\linewidth}{Toolset for performance analysis of large-scale parallel applications. \newline {\color{logoblue} \url{http://www.scalasca.org}}} \\ 
scalasca-gnu-mpich-ohpc &
& \\ 
scalasca-gnu-mvapich2-ohpc &
& \\ 
scalasca-gnu-openmpi-ohpc &
& \\ 
scalasca-gnu7-impi-ohpc &
& \\ 
scalasca-gnu7-mpich-ohpc &
& \\ 
scalasca-gnu7-mvapich2-ohpc &
& \\ 
scalasca-gnu7-openmpi-ohpc &
& \\ 
scalasca-gnu7-openmpi3-ohpc &
& \\ 
scalasca-intel-impi-ohpc &
& \\ 
scalasca-intel-mpich-ohpc &
& \\ 
scalasca-intel-mvapich2-ohpc &
& \\ 
scalasca-intel-openmpi-ohpc &
& \\ 
scalasca-intel-openmpi3-ohpc &
& \\ 
\hline
% <-- end entry for scalasca

% <-- begin entry for scorep
\bottomrule
\end{tabularx}
