

The default underlying network type used by \Lustre{} is {\em tcp}. If your
external \Lustre{} file system is to be mounted using a network type other than
{\em tcp}, additional configuration files are necessary to identify the desired
network type. The example below illustrates creation of modprobe configuration files
instructing \Lustre{} to use an \InfiniBand{} network with the \textbf{o2ib} LNET driver
attached to \texttt{ib0}. Note that these modifications are made to both the
{\em master} host and {\em compute} nodes.

%x\clearpage
% begin_ohpc_run
% ohpc_validation_comment Enable o2ib for Lustre
\begin{lstlisting}[language=bash,keywords={},upquote=true]
[sms](*\#*) echo "options lnet networks=o2ib(ib0)" >> /etc/modprobe.d/lustre.conf
[sms](*\#*) psh compute echo "\""options lnet networks=o2ib(ib0)"\"" \>\> /etc/modprobe.d/lustre.conf
\end{lstlisting}
% end_ohpc_run

With the \Lustre{} configuration complete, the client can be mounted on the {\em master}
and {\em compute} hosts as follows:
% begin_ohpc_run
% ohpc_validation_comment mount Lustre client on master and compute
\begin{lstlisting}[language=bash,keywords={},upquote=true]
[sms](*\#*) mkdir /mnt/lustre
[sms](*\#*) mount -t lustre -o localflock ${mgs_fs_name} /mnt/lustre

# Mount on compute nodes
[sms](*\#*) psh compute mount /mnt/lustre
\end{lstlisting}
% ohpc_indent 0
% ohpc_command fi
% ohpc_validation_newline
% end_ohpc_run
