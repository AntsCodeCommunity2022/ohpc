This guide presents a simple cluster installation procedure using components
from the Forest Peak (\FSP{}) software stack. \FSP{} represents an aggregation
of a number of common ingredients required to deploy and manage an HPC Linux*
cluster including provisioning tools, resource management, I/O clients,
development tools, and a variety of scientific libraries. These packages have
been pre-built with HPC integration in mind and represent a mix of open-source
components combined with \Intel{} development and analysis tools (e.g. \Intel{}
Parallel Studio XE Cluster Edition).
%This install guide assumes availability of a {\em master} host
%that will have access to common OS repositories and the \FSP{} repository in order
%to facilitate software installs and dependency resolution.  
The documentation herein is intended to be reasonably generic, but uses the
underlying motivation of a small, 4-node cluster install to define a step-by-step
process. Several optional customizations are included and the intent is that
these collective instructions can be modified as needed for local site
customizations. 
%% \subsection{Target Audience}
%% 
%% This guide is targeted at experienced \Linux{} system administrators for HPC
%% environments. Knowledge of software package management, system networking, and
%% PXE booting is assumed.  Command-line input examples are highlighted throughout
%% this guide via the following syntax:
%% 
%% \begin{lstlisting}[language=bash,literate={-}{-}1,keywords={},upquote=true]
%% [master]$ echo "FSP hello world"
%% \end{lstlisting}
%% 
%% Unless specified otherwise, the examples presented herein are executed with
%% elevated (root) privileges. The examples also presume use of the BASH login
%% shell, though the equivilent commands in other shells should work fine.
