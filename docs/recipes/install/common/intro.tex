This guide presents a simple cluster installation procedure using components
from the Forest Peak (\FSP{}) software stack. \FSP{} represents an aggregation of a number of
common ingredients required to deploy and manage an HPC Linux* cluster including
provisioning tools, resource management, I/O clients, development tools, and a variety of
scientific libraries. These packages have been pre-built with HPC integration
in mind and represent a mix of open-source components combined with \Intel{}
development and analysis tools (e.g. \Intel{} Parallel Studio XE Cluster Edition).
%This install guide assumes availablility of a {\em master} host
%that will have access to common OS repositories and the \FSP{} repository in order
%to facilitate software installs and dependency resolution.  
The documentation herein is intended to be reasonably generic, but uses the
underlying motivation of a small cluster install to define a step-by-step
process. Several optional customizations are included and the intent is that
these collective instructions can be modified as needed for local site
customizations. 
%targeted for reproducibility on the
%%%internal {\em Zeus} cluster at Intel, and requires access to RPM repositories that are made
%%%available by the \FSP{} Open Build Service (OBS) instance that is used to build
%%%and package components. 
%%%Note: this process is not meant to dictate a
%%%specific, final \FSP{} implementation or installation direction; instead, it
%%%serves as a learning vehicle and demonstrates the ability to use the 2014 \FSP{}
%%%checkpoint to install a small, working cluster.
