This guide presents a simple cluster installation procedure using components
from the \OHPC{} software stack. \OHPC{} represents an aggregation of a number
of common ingredients required to deploy and manage an HPC Linux* cluster
including provisioning tools, resource management, I/O clients, development
tools, and a variety of scientific libraries. These packages have been
pre-built with HPC integration in mind while conforming to common \Linux{}
distribution standards.
The documentation herein is intended to
be reasonably generic, but uses the underlying motivation of a small, 4-node
\iftoggleverb{isxCATstateful} stateful \else stateless \fi 
cluster installation to define a step-by-step process. Several
optional customizations are included and the intent is that these collective
instructions can be modified as needed for local site customizations.
