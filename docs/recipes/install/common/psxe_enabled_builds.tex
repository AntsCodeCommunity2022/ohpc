In addition to the 3rd party development libraries built using the open source
toolchains mentioned in \S\ref{sec:3rdparty}, \OHPC{} also provides {\em
  optional} compatible builds for use with the compilers and MPI stack included
in newer versions of the \IntelR{}~Parallel Studio XE software suite.  These
packages provide a similar hierarchical user
environment experience as other compiler and MPI families present in \OHPC{}.

To take advantage of the available builds, the Parallel Studio software suite
must be obtained and installed separately. Once installed locally, the \OHPC{}
compatible packages can be installed using standard package manager semantics.
Note that licenses are provided free of charge for many categories of use. In
particular, licenses for compilers and developments tools are provided at no
cost to academic researchers or developers contributing to open-source software
projects. More information on this program can be found at:

\begin{center}
  \href{https://software.intel.com/en-us/qualify-for-free-software}
       {\color{blue}{https://software.intel.com/en-us/qualify-for-free-software}}
\end{center}

\begin{center}
\begin{tcolorbox}[]
As noted in \S\ref{sec:master_customization}, the default installation path for
\OHPC{} (\texttt{/opt/ohpc/pub}) is exported over NFS from the {\em master} to the 
compute nodes, but the Parallel Studio installer defaults to a path of 
\texttt{/opt/intel}. To make the \IntelR{} compilers available to the compute 
nodes one must either customize the Parallel Studio installation path to be 
within \texttt{/opt/ohpc/pub}, or alternatively, add an additional NFS export
for \texttt{/opt/intel} that is mounted on desired compute nodes.
\end{tcolorbox}
\end{center}

\noindent To enable all 3rd party builds available in \OHPC{} that are compatible with
\IntelR{}~Parallel Studio, issue the following:

% begin_ohpc_run
% ohpc_comment_header Install Optional Development Tools for use with Intel Parallel Studio \ref{sec:3rdparty_intel}
% ohpc_command if [[ ${enable_intel_packages} -eq 1 ]];then
% ohpc_indent 5
\begin{lstlisting}[language=bash,keywords={},upquote=true,keepspaces]
# Install OpenHPC compatibility packages (requires prior installation of Parallel Studio)
[sms](*\#*) (*\install*) intel-compilers-devel-ohpc
[sms](*\#*) (*\install*) intel-mpi-devel-ohpc
\end{lstlisting}

% ohpc_command if [[ ${enable_opa} -eq 1 ]];then
% ohpc_indent 10
\begin{lstlisting}[language=bash,keywords={},upquote=true,keepspaces]
# Optionally, choose the Omni-Path enabled build for MVAPICH2. Otherwise, skip to retain IB variant
[sms](*\#*) (*\install*) mvapich2-psm2-intel-ohpc
\end{lstlisting}
% ohpc_indent 5
% ohpc_command fi

\iftoggle{isSLES_ww_slurm_x86}{\clearpage}

\begin{lstlisting}[language=bash,keywords={},upquote=true,keepspaces]
# Install 3rd party libraries/tools meta-packages built with Intel toolchain
[sms](*\#*) (*\install*) ohpc-intel-serial-libs
[sms](*\#*) (*\install*) ohpc-intel-io-libs
[sms](*\#*) (*\install*) ohpc-intel-perf-tools
[sms](*\#*) (*\install*) ohpc-intel-python-libs
[sms](*\#*) (*\install*) ohpc-intel-runtimes
[sms](*\#*) (*\install*) ohpc-intel-mpich-parallel-libs
[sms](*\#*) (*\install*) ohpc-intel-mvapich2-parallel-libs
[sms](*\#*) (*\install*) ohpc-intel-openmpi3-parallel-libs
[sms](*\#*) (*\install*) ohpc-intel-impi-parallel-libs
\end{lstlisting}
% ohpc_indent 0
% ohpc_command fi
% end_ohpc_run

