To add Lustre client support on the cluster, it necessary to install the client
and associated modules on each host needing to access a Lustre file system.  In
this recipe, it is assumed that the Lustre file system is hosted by servers
that are pre-existing and are not part of the install process. Outlining the
variety of Lustre client mounting options is beyond the scope of this document,
%(please consult Lustre documentation for more details on failover configuration
%support and networking options), 
but the general requirement is to add a mount entry for the desired file system
that defines the management server (MGS) and underlying network transport
protocol.  To add client mounts on both the {\em master} server and {\em
compute} image, the following commands can be used. Note that the MGS node is
identified by the \texttt{<mgs\_node>} variable and the external Lustre file
system name is identified by the \texttt{<lustre\_fs\_name>} variable. In this
example, the file system is configurd to be mounted locally
as \texttt{/mnt/lustre}.
