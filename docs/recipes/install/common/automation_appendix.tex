\subsection{Installation Template}  \label{appendix:template_script}

This appendix highlights the availability of a companion installation script
that is included with \OHPC{} documentation.  This script, when combined with
local site inputs, can be used to implement a starting recipe for
bare-metal system installation and configuration. This template script is used
during validation efforts to test cluster installations and is provided as a
convenience for administrators as a starting point for potential site
customization. 

The template script relies on the use of a simple text file to
define local site variables that were outlined in \S\ref{sec:inputs}.
% The collection of command-line instructions that are in this guide, when
% combined with local site inputs,  To aid in direct usage of the
% commands called out in this particular recipe, and to also allow for potential
% site customization, the \OHPC{} documentation package includes a template script
% summarizing the commands used herein. This script can be used in conjunction
% with a simple text file to define the local site variables defined in the
% previous section (
By default, the template install script attempts to use local variable settings
sourced from the \path{/opt/fsp/pub/doc/recipes/vanilla/input.local} file,
however, this choice can be overridden by the use of the
\texttt{FSP\_INPUT\_LOCAL} environment variable. The template install script is
intended for execution on the SMS {\em master} host and is installed as part of
the \texttt{docs-fsp} package into \path{/opt/fsp/pub/doc/recipes/vanilla/recipe.sh}.
After enabling the \OHPC{} repository and reviewing the guide for additional information on the intent of the
commands, the general starting approach for using this template is as follows:

\begin{enumerate}
\item Install the \texttt{docs-fsp} package

\begin{lstlisting}[language=bash,keywords={}]
[master]$ (*\install*) docs-fsp
\end{lstlisting}

\item Copy the provided template input file to use as a starting point to
  define local site settings:
\begin{lstlisting}
[master]$ cp  /opt/fsp/pub/doc/recipes/vanilla/input.local input.local
\end{lstlisting}

\item Update \path{input.local} with desired settings

\item Copy the template installation script which contains command-line
  instructions culled from this guide.

\begin{lstlisting}[language=bash,keywords={}]
[master]$ cp -p /opt/fsp/pub/doc/recipes/vanilla/recipe.sh .
\end{lstlisting}

\item Review and edit \path{recipe.sh} to suite.

\item Use environment variable to define local input file and execute
  \path{recipe.sh} to perform a local installation.

\begin{lstlisting}[language=bash,keywords={}]
[master]$ export OHPC_INPUT_LOCAL=./input.local
[master]$ ./recipe.sh
\end{lstlisting}
\end{enumerate}



