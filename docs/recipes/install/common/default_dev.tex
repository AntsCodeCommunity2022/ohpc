System users often find it convenient to have a default development environment
in place so that compilation can be performed directly for parallel programs
requiring MPI. This setup can be conveniently enabled via modules and the \OHPC{}
modules environment is pre-configured to load an \texttt{ohpc} module on login
(if present). The following package install provides a default
environment that enables autotools, the \GNU{} compiler toolchain, and the
MVAPICH2 MPI stack.

% begin_ohpc_run
\begin{lstlisting}[language=bash]
[sms](*\#*) (*\install*) lmod-defaults-gnu-mvapich2-ohpc
\end{lstlisting}
% end_ohpc_run

\begin{center}
\begin{tcolorbox}[]
\small
If you want to change the default environment from the suggestion above, \OHPC{}
also provides the \GNU{} compiler toolchain with the OpenMPI stack:
\begin{itemize*}
\item lmod-defaults-gnu-openmpi-ohpc
\item lmod-defaults-gnu-mpich-ohpc
\end{itemize*}
\end{tcolorbox}
\end{center}
