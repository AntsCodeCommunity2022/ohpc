\noindent Similarly, to import the global Slurm configuration file and the cryptographic
key and associated file permissions that are required by the {\em munge}
authentication library to be available on every host in the resource management
pool, issue the following:

% begin_ohpc_run
\begin{lstlisting}[language=bash,literate={-}{-}1,keywords={},upquote=true]
[sms](*\#*) echo "/etc/slurm/slurm.conf -> /etc/slurm/slurm.conf" >> /install/custom/netboot/compute.synclist
[sms](*\#*) echo "/etc/munge/munge.key -> /etc/munge/munge.key" >>/install/custom/netboot/compute.synclist
[sms](*\#*) echo "/etc/munge/ -> /etc/munge/" >> /install/custom/netboot/compute.synclist
[sms](*\#*) echo "/var/log/munge -> /var/log/munge" >> /install/custom/netboot/compute.synclist
[sms](*\#*) echo "/var/lib/munge -> /var/lib/munge" >> /install/custom/netboot/compute.synclist
\end{lstlisting}
% \end_ohpc_run

\begin{center}
\begin{tcolorbox}[]
\small
The ``\texttt{updatenode compute -F}'' command can be used to distribute changes made to any
defined synchronization files on the SMS host. Users wishing to automate this process may
want to consider adding a crontab entry to perform this action at defined intervals.
\end{tcolorbox}
\end{center}
