To begin, enable use of the \OHPC{} repository by adding it to the local list
of available package repositories. Note that this requires network access from
your {\em master} server to the \OHPC{} repository, or alternatively, that
the \OHPC{} repository be mirrored locally. The example which follows
illustrates the addition of the \OHPC{} repository using
the \texttt{\$\{ohpc\_repo\}} variable.

\begin{lstlisting}[language=bash,keywords={},basicstyle=\fontencoding{T1}\footnotesize\ttfamily,
	literate={VER}{\OHPCVersion{}}1 {OSREPO}{\OSRepo{}}1 {-}{-}1]
[sms](*\#*) ohpc_repo=http://build.openhpc.community/OpenHPC:/VER/OSREPO/OpenHPC:VER.repo
\end{lstlisting}

\noindent Once set, the repository can be enabled via the following: 






