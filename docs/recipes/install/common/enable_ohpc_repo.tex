To begin, enable use of the \OHPC{} repository by adding it to the local list
of available package repositories. Note that this requires network access from
your {\em master} server to the \OHPC{} repository, or alternatively, that
the \OHPC{} repository be mirrored locally.  In cases where network external
connectivity is available, \OHPC{} provides an \texttt{ohpc-release} package
that includes GPG keys for package signing and repository enablement.  The
example which follows illustrates installation of the \texttt{ohpc-release}
package directly from the \OHPC{} build server.

% thank you internet 
\newcommand{\iftoggleverb}[1]{%
  \ifcsdef{etb@tgl@#1}
      {\csname etb@tgl@#1\endcsname\iftrue\iffalse}
          {\etb@noglobal\etb@err@notoggle{#1}\iffalse}%
}

\iftoggleverb{isCentOS}
% CentOS
\begin{lstlisting}[language=bash,keywords={},basicstyle=\fontencoding{T1}\fontsize{7.9}{10}\ttfamily,
%\begin{lstlisting}[language=bash,keywords={},basicstyle=\ttfamily\fontsize{7.5}{10}\ttfamily,
	literate={VER}{\OHPCVerTree{}}1 {OSREPO}{\OSRepo{}}1 {-}{-}1]
[sms](*\#*) yum install http://build.openhpc.community/OpenHPC:/VER/OSREPO/x86_64/ohpc-release-VER-1.x86_64.rpm
\end{lstlisting}
\else
% non-CentOS
\begin{lstlisting}[language=bash,keywords={},basicstyle=\fontencoding{T1}\footnotesize\ttfamily,
	literate={VER}{\OHPCVerTree{}}1 {OSREPO}{\OSRepo{}}1 {-}{-}1]
[sms](*\#*) rpm -ivh http://build.openhpc.community/OpenHPC:/VER/OSREPO/x86_64/ohpc-release-VER-1.x86_64.rpm
\end{lstlisting}
\fi







