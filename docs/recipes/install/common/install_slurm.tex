The following command adds the \SLURM{} workload manager server components to the
chosen {\em master} host. Note that client-side components will be added to
the corresponding compute image in a subsequent step.

% begin_fsp_run
% fsp_validation_comment Add resource manager
\begin{lstlisting}[language=bash,keywords={}]
[master]$ (*\groupinstall*) fsp-slurm-server
\end{lstlisting}
% end_fsp_run

\SLURM{} requires the delineation of the system user that runs the underlying
resource management daemons. The default configuration file that is supplied
with the \FSP{} build of \SLURM{} identifies this \texttt{SlurmUser} to be a
dedicated user named \texttt{slurm} and this user must exist. 
The following command can be used to add this user to the {\em
  master} server:

% begin_fsp_run
% fsp_validation_comment Add slurm user
\begin{lstlisting}[language=bash,keywords={}]
[master]$ useradd slurm
\end{lstlisting}
% end_fsp_run

To facilitate running test jobs later on under the auspices of a resource
manager, we also add a dedicated ``test'' user as follows:

% begin_fsp_run
% fsp_validation_comment Add test user
\begin{lstlisting}[language=bash,keywords={}]
[master]$ useradd -m test
\end{lstlisting}
% end_fsp_run
