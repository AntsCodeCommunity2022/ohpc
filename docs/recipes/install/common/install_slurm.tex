OpenHPC provides multiple options for distributed resource management. 
The following command adds the \SLURM{} workload manager server components to the
chosen {\em master} host. Note that client-side components will be added to
the corresponding compute image in a subsequent step.

% begin_ohpc_run
% ohpc_comment_header Add resource management services on master node \ref{sec:add_rm}
\begin{lstlisting}[language=bash,keywords={}]
[sms](*\#*) (*\groupinstall*) ohpc-slurm-server
\end{lstlisting}
% end_ohpc_run

\SLURM{} requires the designation of a system user that runs the underlying
resource management daemons. The default configuration file that is supplied
with the \OHPC{} build of \SLURM{} identifies this \texttt{SlurmUser} to be a
dedicated user named ``slurm'' and this user must exist. 
The following command can be used to add this user to the {\em
  master} server:

% begin_ohpc_run
\begin{lstlisting}[language=bash,keywords={}]
[sms](*\#*) useradd slurm
\end{lstlisting}
% end_ohpc_run
 
 Other versions of this guide are available that describe installation of other 
 resource management systems, and they can be found in the \texttt{docs-ohpc}
 package.
