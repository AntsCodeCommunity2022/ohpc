Warewulf normally defaults to running the assembled VNFS image out of system
memory in a {\em stateless} configuration. Alternatively, Warewulf can also be
used to partition and format persistent storage such that the VNFS image can be
installed locally to disk in a {\em stateful} manner.  This does, however,
require that a boot loader (GRUB) be added to the image as follows:

\begin{lstlisting}[language=bash,literate={-}{-}1,keywords={},upquote=true]
# Add GRUB2 bootloader and re-assemble VNFS image
[sms](*\#*) (*\chrootinstall*) grub2
[sms](*\#*) wwvnfs  -y --chroot $CHROOT
\end{lstlisting}

\noindent Enabling stateful nodes also requires additional site-specific, disk-related
parameters in the Warewulf configuration. In the example that follows, the
compute node configuration is updated to define where to install the GRUB
bootloader, which disk to partition, which partition to format, and what the
filesystem layout will look like.

\begin{lstlisting}[language=bash,literate={-}{-}1,keywords={},upquote=true]
# Update node object parameters
[sms](*\#*) export filesystems="mountpoint=/boot:dev=sda1:type=ext3:size=500," \
              "dev=sda2:type=swap:size=32768," \
              "mountpoint=/:dev=sda3:type=ext3:size=fill"
[sms](*\#*) for ((i=0; i<$num_computes; i++)) ; do 
              wwsh -y object modify -s bootloader=sda ${c_name[$i]};
              wwsh -y object modify -s diskpartition=sda ${c_name[$i]};
              wwsh -y object modify -s diskformat=sda1,sda2,sda3 ${c_name[$i]};
              wwsh -y object modify -s filesystems="$filesystems" ${c_name[$i]};
        done
\end{lstlisting}

\noindent Upon subsequent reboot of the modified nodes, Warewulf will partition
and format the disk to host the desired VNFS image.  Once installed to disk,
Warewulf can be instructed to subsequently boot from local storage
(alternatively, the BIOS boot option order could be updated to reflect a desire
to boot from disk):

\begin{lstlisting}[language=bash,literate={-}{-}1,keywords={},upquote=true]
# Update node object parameters
[sms](*\#*) for ((i=0; i<$num_computes; i++)) ; do 
              wwsh -y object modify -s bootlocal=EXIT ${c_name[$i]};
        done
\end{lstlisting}


\noindent Deleting the bootlocal object parameter will cause Warewulf to once
again reformat and re-install to local storage upon a new PXE boot request.
