Warewulf normally defaults to running the assembled VNFS image out of system
memory in a {\em stateless} configuration. Alternatively, Warewulf can also be
used to partition and format persistent storage such that the VNFS image can be
installed locally to disk in a {\em stateful} manner.  This does, however,
require that a boot loader (GRUB) be added to the image as follows:

\begin{lstlisting}[language=bash,literate={-}{-}1,keywords={},upquote=true]
# Add GRUB2 bootloader and re-assemble VNFS image
[sms](*\#*) (*\chrootinstall*) grub2
[sms](*\#*) wwvnfs  --chroot $CHROOT
\end{lstlisting}

\noindent Enabling stateful nodes also requires additional site-specific, disk-related
parameters in the Warewulf configuration, and several example partitioning scripts are 
provided in the distribution. 

\begin{lstlisting}[language=bash,literate={-}{-}1,keywords={},upquote=true]
# Select (and customize) appropriate parted layout example
[sms](*\#*) export PARTED_CMDS=/etc/warewulf/filesystem/examples/gpt_example.cmds
[sms](*\#*) cp /etc/warewulf/filesystem/examples/gpt_example.cmds /etc/warewulf/filesystem/gpt.cmds
[sms](*\#*) wwsh provision set --filesystem=gpt "${compute_regex}" 
\end{lstlisting}

\noindent Upon subsequent reboot of the modified nodes, Warewulf will partition
and format the disk to host the desired VNFS image.  Once installed to disk,
Warewulf can be instructed to subsequently boot from local storage
(alternatively, the BIOS boot option order could be updated to reflect a desire
to boot from disk):

\begin{lstlisting}[language=bash,literate={-}{-}1,keywords={},upquote=true]
# After provisioning, update node object parameters to boot from local storage
[sms](*\#*) wwsh -y object modify -s bootlocal=EXIT -t node "${compute_regex}"
\end{lstlisting}


\noindent Deleting the bootlocal object parameter will cause Warewulf to once
again reformat and re-install to local storage upon a new PXE boot request.
