\subsection{Inputs}
As this recipe details installing a cluster
starting from bare-metal, there is a requirement to define IP addresses and gather
hardware MAC addresses in order to support a controlled provisioning process. These values
are necessarily unique to the hardware being used, and this document uses \texttt{<variable>}
names in the command-line examples that follow to highlight where local site
inputs are required. A summary of the required variables used in this recipe
are as follows: \\

\vspace*{0.2cm}
\begin{tabular}{@{}>{\textbullet}cll@{}}
& \texttt{<master\_name>}  & {\small \# Hostname for master server} \\
& \texttt{<master\_ip>} & {\small \# Local private IP address on master server} \\
& \texttt{<sms\_eth\_internal>} & {\small \# Internal ethernet interface on SMS} \\
& \texttt{<internal\_netmask>} & {\small \# Subnet netmask for internal cluster network} \\
& \texttt{<bmc\_username>} & {\small \# BMC username for use by IPMI} \\
& \texttt{<bmc\_password>} & {\small \# BMC password for use by IPMI} \\
& \texttt{<c1\_ip>},\texttt{<c2\_ip>},\texttt{<c3\_ip>},\texttt{<c4\_ip>}
& {\small \# Desired compute node addresses} \\
& \texttt{<c1\_bmc>},\texttt{<c2\_bmc>},\texttt{<c3\_bmc>},\texttt{<c4\_bmc>}
& {\small \# BMC addresses for computes} \\
& \texttt{<c1\_mac>},\texttt{<c2\_mac>},\texttt{<c3\_mac>},\texttt{<c4\_mac>}
& {\small \# MAC addresses for computes} \\
& \texttt{<mgs\_fs\_name>} & {\small \# Lustre MGS mount name (\em Optional)} \\
& \texttt{<master\_ipoib>} & {\small \# IPoIB address for master server (\em Optional)} \\
& \texttt{<ipoib\_netmask>} & {\small \# Subnet netmask for internal IPoIB (\em Optional)} \\
& \texttt{<c1\_ipoib>},\texttt{<c2\_ipoib>},\texttt{<c3\_ipoib>},\texttt{<c4\_ipoib>}
& {\small \# IPoIB addresses for computes (\em Optional)}\\

\end{tabular}
