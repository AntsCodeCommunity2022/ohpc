\begin{center}
\begin{tcolorbox}[]
\small \Warewulf{} includes a utility named \texttt{wwnodescan} 
to automatically register new compute nodes versus the approach outlined
above which requires hardware MAC addresses to be gathered in advance.  With
\texttt{wwnodescan}, nodes will be added to the \Warewulf{} database in the
order in which their DHCP requests are received by the master, so care must be
taken to boot nodes in the order one wishes to see preserved in the Warewulf
database. The IP address provided will be incremented after each node is found,
and the utility will exit after all specified nodes have been found. Example
usage is highlighted below:
\begin{lstlisting}[language=bash,keywords={},upquote=true,basicstyle=\footnotesize\ttfamily,literate={BOSVER}{\baseos{}}1]
[sms](*\#*) wwnodescan --netdev=${eth_provision} --ipaddr=${c_ip[0]} --netmask=${internal_netmask} \
    --vnfs=BOSVER --bootstrap=`uname -r` ${c_name[0]}-${c_name[3]}
\end{lstlisting}
\end{tcolorbox}
\end{center}
