With the provisioning services enabled, the next step is to define and
customize a system image that can subsequently be used to provision one or more
{\em compute} nodes. The following subsections highlight this process.

\subsubsection{Build initial BOS image} \label{sec:assemble_bos}

The \OHPC{} build of \Warewulf{} includes specific enhancements enabling support for
\baseOS{}. The following steps illustrate the process to build a minimal, default
image for use with \Warewulf{}. We begin by creating a directory structure on the 
{\em master} host that will represent the root filesystem of the compute node. The 
default location for this example is in
\texttt{/opt/ohpc/admin/images/\baseos{}}.

\begin{center}
  \begin{tcolorbox}[]
    \small Note that \Warewulf{} requires access to a zypper repository during the 
    \texttt{wwmkchroot} process. The easiest way to provide this is to mount 
    the SUSE-SLE-12-SP4-Server ISO image on a web server that can communicate 
    with the {\em master} host, set the \texttt{\$\{BOS\_MIRROR\}} environment 
    variable to that URL, and update the template file {\em prior} to running 
    the \texttt{wwmkchroot} command below. For example:

% begin_ohpc_run
% ohpc_command if [ ! -z ${BOS_MIRROR+x} ]; then
% ohpc_indent 5
\begin{lstlisting}[language=bash,keywords={}]
# Override default OS repository - set BOS_MIRROR variable to desired repo location
[sms](*\#*) perl -pi -e "s#^ZYP_MIRROR=(\S+)#ZYP_MIRROR=${BOS_MIRROR}#" \
   /usr/lib/warewulf/wwmkchroot/sles-12.tmpl
\end{lstlisting}
% ohpc_indent 0
% ohpc_command fi
% end_ohpc_run

\end{tcolorbox}
\end{center}

\iftoggle{isSLES_ww_pbs_aarch}{\vspace*{-.5cm}}

% begin_ohpc_run
% ohpc_comment_header Create compute image for Warewulf \ref{sec:assemble_bos}
\begin{lstlisting}[language=bash,literate={-}{-}1,keywords={},upquote=true,keepspaces,literate={BOSVER}{\baseos{}}1]
# Define chroot location
[sms](*\#*) export CHROOT=/opt/ohpc/admin/images/BOSVER
# Build initial chroot image
[sms](*\#*) mkdir -p -m 755 $CHROOT                        # create chroot housing dir
[sms](*\#*) mkdir -m 755 $CHROOT/dev                       # create chroot /dev dir
[sms](*\#*) mknod -m 666 $CHROOT/dev/zero c 1 5            # create /dev/zero device
[sms](*\#*) wwmkchroot -v opensuse-15.1 $CHROOT            # create base image
# Enable OpenHPC repo in chroot
[sms](*\#*) cp -p /etc/zypp/repos.d/OpenHPC.repo $CHROOT/etc/zypp/repos.d
# Import GPG keys for chroot repository usage
[sms](*\#*) zypper -n --root $CHROOT --no-gpg-checks --gpg-auto-import-keys refresh
\end{lstlisting}
% end_ohpc_run
