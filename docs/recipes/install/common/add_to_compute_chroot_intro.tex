The \texttt{wwmkchroot} process used in the previous step is designed to
provide a minimal \baseOS{} configuration. Next, we add additional components
to include resource management client services, NTP support, and
other additional packages to support the default \OHPC{} environment. This
process augments the chroot-based install performed by \texttt{wwmkchroot} to
modify the base provisioning image and will access the BOS and \OHPC{}
repositories to resolve package install requests. We begin by installing a few
common base packages:

% begin_ohpc_run
% ohpc_comment_header Add OpenHPC base components to compute image \ref{sec:add_components}
\begin{lstlisting}[language=bash,literate={-}{-}1,keywords={},upquote=true]
# Install compute node base meta-package
[sms](*\#*) (*\chrootinstall*) ohpc-base-compute
\end{lstlisting}
% end_ohpc_run

To access the remote
repositories by hostname (and not IP addresses), the chroot environment needs
to be updated to enable DNS resolution. Assuming that the {\em master} host has
a working DNS configuration in place, the chroot environment can be updated
with a copy of the configuration as follows:

% begin_ohpc_run
% ohpc_comment_header Add OpenHPC components to compute image \ref{sec:add_components}
\begin{lstlisting}[language=bash,literate={-}{-}1,keywords={},upquote=true]
[sms](*\#*) cp -p /etc/resolv.conf $CHROOT/etc/resolv.conf
\end{lstlisting}
% end_ohpc_run

\noindent Now, we can include additional components to the compute instance using
\texttt{\pkgmgr{}} to install into the chroot location defined previously:
