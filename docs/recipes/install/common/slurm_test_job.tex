With the resource manager enabled for production usage, users should now be
able to run jobs. To demonstrate this, we will add a ``test'' user on the {\em master}
host that can be used to run an example job.

% begin_ohpc_run
\begin{lstlisting}[language=bash,keywords={}]
[sms](*\#*) useradd -m test
\end{lstlisting}
% end_ohpc_run

\Warewulf{} installs a utility on the compute nodes to automatically 
synchronize known files from the provisioning server at five minute intervals. In this
recipe, recall that we previously registered credential files with Warewulf (e.g. passwd,
group, and shadow) so that these files would be propagated during compute node
imaging. However, with the addition of a new ``test'' user above, the files
have been outdated and we need to update the Warewulf database to incorporate
the additions. This re-sync process can be accomplished as follows:

% begin_ohpc_run
\begin{lstlisting}[language=bash,keywords={}]
[sms](*\#*) wwsh file resync passwd shadow group
\end{lstlisting}
% end_ohpc_run

\begin{center}
\begin{tcolorbox}[]
\small
After re-syncing to notify Warewulf of file modifications made on the {\em
master} host, it should take approximately 5 minutes for the changes to
propagate. However, you can also manually pull the changes from compute nodes
via the following:
% begin_ohpc_run
\begin{lstlisting}[language=bash,keywords={}]
[sms](*\#*) pdsh -w c[1-4] /warewulf/bin/wwgetfiles 
\end{lstlisting}
% end_ohpc_run
\end{tcolorbox}
\end{center}

%We must first update the database with the change we just made to passwd and 
%shadow files, and since we are impatient we will trigger the compute node pull.

\OHPC{} includes a simple ``hello-world'' MPI application in the
\path{/opt/ohpc/pub/examples} directory that can be used for this quick
compilation and execution. \OHPC{} also provides a companion job-launch script
named \texttt{prun} that is installed in concert with the pre-packaged MPI
toolchains. At present, \OHPC{} is unable to include the PMI process management
server normally included within \SLURM{} which implies that \texttt{srun} cannot
be use for MPI job launch. Instead, native job launch mechanisms provided by
the MPI stacks are utilized and \texttt{prun} abstracts this process for the
various stacks to retain a single launch command.

\nottoggle{isCentOS}{\clearpage}

  
\subsection{Interactive execution}
To use the newly created ``test'' account to compile and execute the
application {\em interactively} through the resource manager, execute the
following (note the use of \texttt{prun} for parallel job launch which summarizes
the underlying native job launch mechanism being used):

\begin{lstlisting}[language=bash,keywords={}]
# switch to "test" user
[sms](*\#*) su - test

# Compile MPI "hello world" example
[test@sms ~]$ mpicc -O3 /opt/ohpc/pub/examples/mpi/hello.c

# Submit interactive job request and use prun to launch executable
[test@sms ~]$ srun -n 8 -N 2 --pty /bin/bash

[test@c1 ~]$ prun ./a.out

[prun] Master compute host = c1
[prun] Launch cmd = mpiexec.hydra -bootstrap slurm ./a.out

 Hello, world (8 procs total)
    --> Process #   0 of   8 is alive. -> c1
    --> Process #   4 of   8 is alive. -> c2
    --> Process #   1 of   8 is alive. -> c1
    --> Process #   5 of   8 is alive. -> c2
    --> Process #   2 of   8 is alive. -> c1
    --> Process #   6 of   8 is alive. -> c2
    --> Process #   3 of   8 is alive. -> c1
    --> Process #   7 of   8 is alive. -> c2
\end{lstlisting}

\begin{center}
\begin{tcolorbox}[]
The following table provides approximate equivalences between slurm and PBS
Pro:
\vspace*{0.15cm}
\input common/rms_equivalence_table
\end{tcolorbox}
\end{center}
\nottoggle{isCentOS}{\clearpage}

\iftoggle{isCentOS}{\clearpage}

\subsection{Batch execution}

For batch execution, \OHPC{} provides a simple job script for reference (also
housed in the \path{/opt/ohpc/pub/examples} directory. This example script can
be used as a starting point for submitting batch jobs to the resource manager
and the example below illustrates use of the script to submit a batch job for
execution using the same executable referenced in the previous interactive example.
%Alternatively, a simple batch script can be written and submitted to the
%scheduler using sbatch to run when nodes become available. 

\begin{lstlisting}[language=bash,keywords={}]
# copy example job script
[test@sms ~]$ cp /opt/ohpc/pub/examples/slurm/job.mpi .

# examine contents (and edit to set desired job sizing characteristics)
[test@sms ~]$ cat job.mpi
#!/bin/bash

#SBATCH -J test               # Job name
#SBATCH -o job.%j.out         # Name of stdout output file (%j expands to %jobId)
#SBATCH -N 2                  # Total number of nodes requested
#SBATCH -n 16                 # Total number of mpi tasks #requested
#SBATCH -t 01:30:00           # Run time (hh:mm:ss) - 1.5 hours

# Launch MPI-based executable

prun ./a.out

# Submit job for batch execution
[test@sms ~]$ sbatch job.mpi
Submitted batch job 339
\end{lstlisting}

\begin{center}
\begin{tcolorbox}[]
\small
The use of the \texttt{\%j} option in the example batch job script shown is a convenient
way to track application output on an individual job basis. The \texttt{\%j} token
is replaced with the \SLURM{} job allocation number once assigned (job~\#339 in
this example).
\end{tcolorbox}
\end{center}


