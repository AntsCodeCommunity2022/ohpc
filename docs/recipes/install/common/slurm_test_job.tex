With the resource manager enabled for production usage, users should now be
able to run jobs.  Recall that we added a ``test'' user on the {\em master}
host in \S\ref{sec:add_rm} that can now be used to run an example test job.
\FSP{} includes a simple ``hello-world'' MPI application in the
\texttt{/opt/fsp/pub/examples} directory that can be used for this quick
compilation and execution.  To use the test account to compile and execute the
application interactively through the resource manager, execute the following:

\begin{lstlisting}[language=bash,keywords={}]
# switch to "test" user
[master]$ su - test

# Compile MPI "hello world" example
[test@master ~]$ mpicc -o hello -O3 /opt/fsp/pub/examples/mpi/hello.c

# Submit interactive job request and use prun to launch executable
[test@master ~]$ srun -n 8 -N 2 --pty /bin/bash

[test@c1 ~]$ prun ./hello

[prun] Master compute host = c1
[prun] Launch cmd = mpiexec.hydra -bootstrap slurm ./hello

 Hello, world (8 procs total)
    --> Process #   0 of   8 is alive. -> c1
    --> Process #   4 of   8 is alive. -> c2
    --> Process #   1 of   8 is alive. -> c1
    --> Process #   5 of   8 is alive. -> c2
    --> Process #   2 of   8 is alive. -> c1
    --> Process #   6 of   8 is alive. -> c2
    --> Process #   3 of   8 is alive. -> c1
    --> Process #   7 of   8 is alive. -> c2
\end{lstlisting}
