\OHPC{} includes a monitoring capability that can optionally be installed to
support unified data collection across the cluster. This monitoring system is
based on \ORCM{}, the open resilient cluster manager project, and provides a
convenient way to aggregate key node metrics including environmental sensors,
processor power, and resource utilization into a centralized database for
historical logging and analysis.  In a typical HPC configuration, \ORCM{} data
collection relies on having IPMI access to the baseboard management controller
(BMC) on each back-end compute resource. The data collection itself can either
be performed using {\em in-band} methods where a daemon is instantiated on each
compute resource or via {\em out-of-band} methods using external IPMI access on
a management network. To include all the components necessary to enable an \ORCM{}
server running on the {\em master} host, issue the following:

% begin_ohpc_run
% ohpc_validation_comment Install ORCM server
\begin{lstlisting}[language=bash]
[master](*\#*) (*\groupinstall*) ohpc-orcm-server
\end{lstlisting}
% end_ohpc_run

In order to coalesce sensor data into a central repository, \ORCM{} requires
data {\em aggregators} that are responsible for collecting data from one or
more monitored resources. For a small cluster, a common configuration is to
have a single aggregator defined and, in this example, we choose the {\em master} host
as the data aggregator. In order to store historical logging data, \ORCM{}
includes support to store aggregated data into a centralized database. In this
example, we will create an \ORCM{} database using {\em postgres} on the chosen
master host. The necessary configuration file changes and commands to create the
underlying database for \ORCM{} are highlighted below. Note that a generic DB
password of ``orcmpassword'' is used; this should be altered locally to a
site-specific credential. If the DB password is changed, a corresponding update
for the new credential should also be placed in the final
\texttt{/etc/sysconfig/orcmd} file.
