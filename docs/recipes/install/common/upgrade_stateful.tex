\subsection{Upgrading OpenHPC Packages}  \label{appendix:upgrade}


As newer \OHPC{} releases are made available, users are encouraged to upgrade
their locally installed packages against the latest repository versions to
obtain access to bug fixes and newer component versions. This can be
accomplished with the underlying package manager as \OHPC{} packaging maintains
versioning state across releases. Also, package builds available from the
\OHPC{} repositories have ``\texttt{-ohpc}'' appended to their names so that
wild cards can be used as a simple way to obtain updates. The following general
procedure highlights a method for upgrading existing installations.
When upgrading from a minor release older than v\OHPCVerTree{}, you will first
need to update your local \OHPC{} repository configuration to point against the
v\OHPCVerTree{} release (or update your locally hosted mirror). Refer to
\S\ref{sec:enable_repo} for more details on enabling the latest
repository. In contrast, when upgrading between micro releases on the same
branch (e.g. from v\OHPCVerTree{} to \OHPCVerTree{}.2), there is no need to
adjust local package manager configurations when using the public repository as
rolling updates are pre-configured.
 
The initial step, when using a local mirror as described in
\S\ref{sec:enable_repo}, is downloading a new tarball from \texttt{http://build.openhpc.community/dist}. 
After this, move (or remove) the old repository directory, unpack
the new tarball and run the \texttt{make\_repo.sh} script. In the following,
substitute \texttt{x.x.x} with the desired version number. 

\begin{lstlisting}[language=bash,keywords={},basicstyle=\fontencoding{T1}\fontsize{7.6}{10}\ttfamily,
	literate={VER}{\OHPCVerTree{}}1 {OSREPO}{\OSTree{}}1 {TAG}{\OSTag{}}1 {ARCH}{\arch{}}1 {-}{-}1 
        {VER2}{\OHPCVersion{}}1]
[sms](*\#*) wget http://repos.openhpc.community/dist/x.x.x/OpenHPC-x.x.x.OSREPO.ARCH.tar
[sms](*\#*) mv $ohpc_repo_dir "$ohpc_repo_dir"_old
[sms](*\#*) mkdir -p $ohpc_repo_dir
[sms](*\#*) tar xvf OpenHPC-x.x.x.OSREPO.ARCH.tar -C $ohpc_repo_dir
[sms](*\#*) $ohpc_repo_dir/make_repo.sh
\end{lstlisting}

The {\em compute} hosts will pickup new packages automatically as long as the
repository location on the SMS stays the same. If you prefer to use a different
location, setup repositories on the nodes as described in
\S\ref{sec:enable_repo}.

The actual update is performed as follows:
\begin{enumerate*}
\item (Optional) Ensure repo metadata is current (on head and compute nodes.)
  Package managers will naturally do this on their own over time,
  but if you are wanting to access updates immediately after a new release,
  the following can be used to sync to the latest.

\begin{lstlisting}[language=bash,keywords={}]
[sms](*\#*)  (*\clean*)
[sms](*\#*)  psh compute (*\clean*)
\end{lstlisting}

\item Upgrade master (SMS) node

\begin{lstlisting}[language=bash,keywords={}]
[sms](*\#*)  (*\upgrade*) "*-ohpc"
\end{lstlisting}
  
\item Upgrade packages on compute nodes

\begin{lstlisting}[language=bash,keywords={}]
[sms](*\#*) psh compute  (*\upgrade*) "*-ohpc"
\end{lstlisting}
  
\nottoggle{isxCATstateful}{\item Rebuild image(s)

\iftoggleverb{isWarewulf}
\begin{lstlisting}[language=bash,keywords={}]
[sms](*\#*) wwvnfs --chroot $CHROOT
\end{lstlisting}
\fi

\iftoggleverb{isxCAT}
\begin{lstlisting}[language=bash,keywords={},basicstyle=\fontencoding{T1}\fontsize{8.0}{10}\ttfamily,
    literate={-}{-}1 {BOSVER}{\baseos{}}1 {ARCH}{\arch{}}1]
[sms](*\#*) packimage BOSVER-x86_64-netboot-compute
\end{lstlisting}
\fi
}

\end{enumerate*}

\noindent Note that to update running services such as a resource manager a
service restart (or a full reboot) is required.

%%% \subsubsection{New component variants}
%%% 
%%% As newer variants of key compiler/MPI stacks are released, \OHPC{} will
%%% periodically add toolchains enabling the latest variant. To stay consistent
%%% throughout the build hierarchy, minimize recompilation requirements for existing
%%% binaries, and allow for multiple variants to coexist, unique delimiters are
%%% used to distinguish RPM package names and module hierarchy.
%%% 
%%% In the case of a fresh install, \OHPC{} recipes default to installation of the
%%% latest toolchains available in a given release branch. However, if upgrading a
%%% previously installed system, administrators can {\em opt-in} to enable new
%%% variants as they become available. To illustrate this point, consider the
%%% previous \OHPC{} 1.3.2 release as an example which contained an {``openmpi''}
%%% MPI variant providing OpenMPI 1.10.x along with runtimes and libraries compiled
%%% with this toolchain. That release also contained the {``llvm4''} compiler
%%% family which has been updated to {``llvm5''} in \OHPC{} 1.3.3.  In the case
%%% where an admin would like to enable the newer {``openmpi3''} toolchain,
%%% installation of these additions is simplified with the use of \OHPC{}'s
%%% meta-packages (see Table~\ref{table:groups} in Appendix
%%% \ref{appendix:manifest}).  The following example illustrates adding the
%%% complete ``openmpi3'' toolchain.  Note that we leverage the convenience
%%% meta-packages containing MPI-dependent builds, and we also update the
%%% modules environment to make it the default.
%%% 
%%% \begin{lstlisting}[language=bash,keywords={}]
%%% # Update default environment
%%% [sms](*\#*) (*\remove*) lmod-defaults-gnu7-openmpi-ohpc
%%% [sms](*\#*) (*\install*) lmod-defaults-gnu7-openmpi3-ohpc
%%% 
%%% # Install OpenMPI 3.x-compiled meta-packages with dependencies
%%% [sms](*\#*)  (*\install*) ohpc-gnu7-perf-tools \
%%%                          ohpc-gnu7-io-libs \
%%%                          ohpc-gnu7-python-libs \
%%%                          ohpc-gnu7-runtimes \
%%%                          ohpc-gnu7-openmpi3-parallel-libs
%%% 
%%% # Install LLVM/Clang 5.x
%%% [sms](*\#*)  (*\install*) llvm5-compilers-ohpc
%%% \end{lstlisting}
%%% 
%%% 
