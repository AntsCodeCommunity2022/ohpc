\clearpage

\subsection{Customization} \label{appendix:customization}

\subsubsection{Adding local Lmod modules to \OHPC{} hierarchy} \label{appendix:modulefiles}
Locally installed applications can easily be integrated in to \OHPC{} systems by
following the Lmod convention laid out by the provided packages. Two sample
module files are included in the \texttt{examples-ohpc} package\textemdash one
representing an application with no compiler or MPI runtime dependencies, and
one dependent on OpenMPI and the \GNU{} toolchain. Simply copy these files to the
prescribed locations, and the \texttt{lmod} application should pick them up
automatically.

\begin{lstlisting}[alsoletter={/,.},morekeywords={example1/1.0, example2/1.0}]
[sms](*\#*) mkdir /opt/ohpc/pub/modulefiles/example1
[sms](*\#*) cp /opt/ohpc/pub/examples/example.modulefile \
    /opt/ohpc/pub/modulefiles/example1/1.0
[sms](*\#*) mkdir /opt/ohpc/pub/moduledeps/gnu7-openmpi/example2
[sms](*\#*) cp /opt/ohpc/pub/examples/example-mpi-dependent.modulefile \
    /opt/ohpc/pub/moduledeps/gnu7-openmpi/example2/1.0
[sms](*\#*) module avail

-------------------------------- /opt/ohpc/pub/moduledeps/gnu7-openmpi ---------------------------------
adios/1.12.0    imb/4.1                 netcdf/4.4.1.1    scalapack/2.0.2    slepc/3.7.4
boost/1.63.0    mpiP/3.4.1              petsc/3.7.6       scalasca/2.3.1     superlu_dist/4.2
example2/1.0    mumps/5.1.1             phdf5/1.10.1      scipy/0.19.1       tau/2.26.1
fftw/3.3.6      netcdf-cxx/4.3.0        pnetcdf/1.8.1     scorep/3.1         trilinos/12.10.1
hypre/2.11.2    netcdf-fortran/4.4.4    ptscotch/6.0.4    sionlib/1.7.1

------------------------------------ /opt/ohpc/pub/moduledeps/gnu7 -------------------------------------
R/3.4.1        metis/5.1.0     ocr/1.0.1              pdtoolkit/3.24    superlu/5.2.1
gsl/2.4        mpich/3.2       openblas/0.2.20        plasma/2.8.0
hdf5/1.10.1    numpy/1.13.1    openmpi/1.10.7  (L)    scotch/6.0.4

-------------------------------------- /opt/ohpc/pub/modulefiles ---------------------------------------
EasyBuild/3.3.1           example1/1.0        ohpc              (L)    valgrind/3.13.0
autotools          (L)    gnu7/7.1.0   (L)    papi/5.5.1
clustershell/1.7.3        hwloc/1.11.7        prun/1.1          (L)
cmake/3.9.1               llvm4/4.0.1         singularity/2.3.1

Where:
L:  Module is loaded

Use "module spider" to find all possible modules.
Use "module keyword key1 key2 ..." to search for all possible modules matching any of the "keys".
\end{lstlisting}

\clearpage
\subsubsection{Rebuilding Packages from Source}  \label{appendix:rpmbuild}

Users of \OHPC{} may find it desirable to rebuild one of the supplied packages
to apply build customizations or satisfy local requirements. One way to
accomplish this is to install the appropriate source RPM, modify the specfile
as needed, and rebuild to obtain an updated binary RPM. A brief example using
the FFTW library is highlighted below.  Note that the source RPMs can be downloaded from the
community build server at \href{https://build.openhpc.community}
{\color{blue}{https://build.openhpc.community}} via a web browser or directly
via \texttt{rpm} as highlighted below. The \OHPC{} build system design
leverages several keywords to control the choice of compiler and MPI families
for relevant development libraries and the \texttt{rpmbuild} example
illustrates how to override the default mpi\_family.

\begin{lstlisting}[language=bash,keywords={},basicstyle=\fontencoding{T1}\footnotesize\ttfamily,
    literate={VER}{\OHPCVerTree{}}1 {OSREPO}{\OSRepo{}}1 {-}{-}1]
# Install rpm-build package from base OS distro
[test@sms ~]$ sudo (*\install*) rpm-build

# Install OpenHPC RPM build macros
[test@sms ~]$ sudo (*\install*) ohpc-buildroot

# Download SRPM from OpenHPC repository and install locally
[test@sms ~]$ rpm -i \
  http://build.openhpc.community/OpenHPC:/VER/OSREPO/src/fftw-gnu-openmpi-ohpc-3.3.4-4.1.src.rpm

# Modify spec file as desired
[test@sms ~]$ cd ~/rpmbuild/SPECS
[test@sms ~rpmbuild/SPECS]$ perl -pi -e "s/enable-static=no/enable-static=yes/" fftw.spec

# Increment RPM release so package manager will see an update
[test@sms ~rpmbuild/SPECS]$ perl -pi -e "s/Release:   4.1/Release:   5.1/" fftw.spec

# Rebuild binary RPM. Note that additional directives can be specified to modify build
[test@sms ~rpmbuild/SPECS]$ rpmbuild -bb --define "mpi_family mvapich2" fftw.spec

# Install the new package
[test@sms ~]$ sudo (*\install*) ~test/rpmbuild/RPMS/x86_64/fftw-gnu-mvapich2-ohpc-3.3.6-5.1.rpm
\end{lstlisting}
