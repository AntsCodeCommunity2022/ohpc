\Ganglia{} is a scalable distributed system monitoring tool for high-performance
computing systems such as clusters and grids. It allows the user to remotely
view live or historical statistics (such as CPU load averages or network
utilization) for all machines running the {\em gmond} daemon. The following 
commands can be used to enable \Ganglia{} to monitor both the {\em master} and 
{\em compute} hosts.

% begin_ohpc_run
% ohpc_validation_newline
% ohpc_command if [[ ${enable_ganglia} -eq 1 ]];then
% ohpc_indent 5
% ohpc_validation_comment Install Ganglia on master
\begin{lstlisting}[language=bash,keywords={},upquote=true]
# Install Ganglia
[sms](*\#*) (*\groupinstall*) ohpc-ganglia
[sms](*\#*) (*\chrootinstall*) ganglia-gmond-ohpc

# Use example configuration script to enable unicast receiver on master host
[sms](*\#*) cp /opt/ohpc/pub/examples/ganglia/gmond.conf /etc/ganglia/gmond.conf
[sms](*\#*) perl -pi -e "s/<sms>/${sms_name}/" /etc/ganglia/gmond.conf

# Add config to compute image and provide gridname 
[sms](*\#*) cp /etc/ganglia/gmond.conf $CHROOT/etc/ganglia/gmond.conf
[sms](*\#*) echo "gridname MySite" >> /etc/ganglia/gmetad.conf

# Start and enable Ganglia services
[sms](*\#*) systemctl enable gmond
[sms](*\#*) systemctl enable gmetad
[sms](*\#*) systemctl start gmond
[sms](*\#*) systemctl start gmetad
[sms](*\#*) chroot $CHROOT systemctl enable gmond

# Restart web server
[sms](*\#*) systemctl try-restart httpd
\end{lstlisting}
% ohpc_indent 0
% ohpc_command fi
% end_ohpc_run

\noindent Once enabled and running, Ganglia should provide access to a web-based
monitoring console on the {\em master} host. Read access to monitoring metrics
will be enabled by default and can be accessed via a web browser. When running
a web browser directly on the {\em master} host, the Ganglia top-level overview
is available
at \href{http://localhost/ganglia}{\color{blue}{http://localhost/ganglia}}.
When accessing remotely, replace {\em localhost} with the chosen name of your
master host (\texttt{\$\{sms\_name\}}).


