% begin_ohpc_run
% ohpc_validation_newline
% ohpc_validation_comment Add hosts to cluster
\begin{lstlisting}[language=bash,keywords={},upquote=true,basicstyle=\footnotesize\ttfamily,]
# Register nodes for PXE provisioning 
[sms](*\#*) for ((i=0; i<$num_computes; i++)) ; do
		mkdef -t node ${c_name[i]} groups=compute,all ip=${c_ip[i]} mac=${c_mac[i]} netboot=pxe
        done
\end{lstlisting}
% end_ohpc_run

\noindent \xCAT{} requires a network domain name specification for system-wide name
resolution. This value can be set to match your local DNS schema or given a
unique identifier such as ``local''. In this recipe, we leverage the
\texttt{\$domain\_name} variable to define as follows:

% begin_ohpc_run
% ohpc_validation_newline
% ohpc_validation_comment Define local domainname
\begin{lstlisting}[language=bash,keywords={},upquote=true,basicstyle=\footnotesize\ttfamily,literate={BOSVER}{\baseos{}}1]
[sms](*\#*) chdef -t site domain=${domain_name}
\end{lstlisting}

%\clearpage
If enabling {\em optional} IPoIB functionality (e.g. to support Lustre over \InfiniBand{}), additional
settings are required to define the IPoIB network with \xCAT{} and specify
desired IP settings for each compute. This can be accomplished as follows for
the {\em ib0} interface:

% begin_ohpc_run
% ohpc_validation_newline
% ohpc_validation_comment Setup IPoIB networking
% ohpc_command if [[ ${enable_ipoib} -eq 1 ]];then
% ohpc_indent 5
\begin{lstlisting}[language=bash,keywords={},upquote=true,basicstyle=\footnotesize\ttfamily]
# Define ib0 netmask
[sms](*\#*) chdef -t network -o ib0 mask=$ipoib_netmask net=${c_ipoib[0]}

# Enable secondary NIC configuration
[sms](*\#*) chdef compute -p postbootscripts=confignics

# Register desired IPoIB IPs per compute
[sms](*\#*) for ((i=0; i<$num_computes; i++)) ; do
		chdef ${c_name[i]} nicips.ib0=${c_ipoib[i]} nictypes.ib0="InfiniBand" nicnetworks.ib0=ib0
        done
\end{lstlisting}
% ohpc_indent 0
% ohpc_command fi
% end_ohpc_run

\clearpage
With the desired compute nodes and domain identified, the remaining steps in the
provisioning configuration process are to define the provisioning mode and
image for the {\em compute} group and use \xCAT{} commands to complete
configuration for network services like DNS and DHCP. These tasks are
accomplished as follows:

%\clearpage
% begin_ohpc_run
% ohpc_validation_newline
% ohpc_validation_comment Complete networking setup, associate provisioning image
\begin{lstlisting}[language=bash,keywords={},upquote=true,basicstyle=\footnotesize\ttfamily,literate={BOSVER}{\baseos{}}1]
# Define provisioning method for computes
[sms](*\#*) chdef -t group compute provmethod=BOSVER-x86_64-netboot-compute

# Complete network service configurations
[sms](*\#*) makehosts
[sms](*\#*) makenetworks
[sms](*\#*) makedhcp -n
[sms](*\#*) makedns -n

# Associate desired provisioning image for computes
[sms](*\#*) nodeset compute osimage=BOSVER-x86_64-netboot-compute

# restart DHCP
[sms](*\#*) service dhcpd restart
\end{lstlisting}

%%% If the Lustre client was enabled for computes in \S\ref{sec:lustre_client}, you
%%% should be able to mount the file system post-boot using the fstab entry
%%% (e.g. via ``\texttt{mount /mnt/lustre}''). Alternatively, if 
%%% you prefer to have the file system mounted automatically at boot time, a simple
%%% postscript can be created and registered with \xCAT{} for this purpose as follows.
%%% 
%%% % begin_ohpc_run
%%% % ohpc_validation_newline
%%% % ohpc_validation_comment Optionally create xCAT postscript to mount Lustre client
%%% % ohpc_command if [ ${enable_lustre_client} -eq 1 ];then
%%% % ohpc_indent 5
%%% \begin{lstlisting}[language=bash,keywords={},upquote=true,basicstyle=\footnotesize\ttfamily,literate={BOSVER}{\baseos{}}1]
%%% # Optionally create postscript to mount Lustre client at boot
%%% [sms](*\#*) echo '#!/bin/bash' > /install/postscripts/lustre-client
%%% [sms](*\#*) echo 'mount /mnt/lustre' >> /install/postscripts/lustre-client
%%% [sms](*\#*) chmod 755 /install/postscripts/lustre-client
%%% # Register script for computes
%%% [sms](*\#*) chdef compute -p postscripts=lustre-client
%%% \end{lstlisting}
%%% % ohpc_indent 0
%%% % ohpc_command fi
%%% % end_ohpc_run
%%% 

