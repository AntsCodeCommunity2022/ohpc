Next, enable use of the public \xCAT{} repository by adding it to the local list
of available package repositories. This also requires network access from
your {\em master} server to the internet, or alternatively, that
the repository be mirrored locally. In this case, we highlight network
enablement by download of the
latest \xCAT{} repo file.

% begin_ohpc_run
% ohpc_validation_newline
% ohpc_comment_header Enable xCAT repositories \ref{sec:enable_xcat}
\begin{lstlisting}[language=bash,keywords={},basicstyle=\fontencoding{T1}\fontsize{8.0}{10}\ttfamily,
	literate={VER}{\OHPCVerTree{}}1 {OSREPO}{\OSTree{}}1 {TAG}{\OSTag{}}1 {ARCH}{\arch{}}1 {-}{-}1]
[sms](*\#*) (*\install*) yum-utils
[sms](*\#*) (*\addrepo*) https://xcat.org/files/xcat/repos/yum/latest/xcat-core/xCAT-core.repo
\end{lstlisting}
% end_ohpc_run

