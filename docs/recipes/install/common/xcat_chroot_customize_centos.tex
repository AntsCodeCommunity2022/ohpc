Prior to assembling the image, it is advantageous to perform any additional
customization within the chroot environment created for the desired {\em
 compute} instance. The following steps document the process identify the
resource manager server, configure NTP for compute resources, and enable \NFS{}
mounting of a \$HOME file system and the public \OHPC{} install path
(\texttt{/opt/ohpc/pub}) that will be hosted by the {\em master} host in this
example configuration.

\vspace*{0.15cm}
% begin_ohpc_run
% ohpc_comment_header Customize system configuration \ref{sec:master_customization}
\begin{lstlisting}[language=bash,literate={-}{-}1,keywords={},upquote=true]
# Add NFS client mounts of /home and /opt/ohpc/pub to base image
[sms](*\#*) echo "${sms_ip}:/home /home nfs nfsvers=3,nodev,nosuid 0 0" >> $CHROOT/etc/fstab
[sms](*\#*) echo "${sms_ip}:/opt/ohpc/pub /opt/ohpc/pub nfs nfsvers=3,nodev 0 0" >> $CHROOT/etc/fstab

# Disable /tftpboot and /install export entries
[sms](*\#*) perl -pi -e "s/tftpboot/#tftpboot/" /etc/exports
[sms](*\#*) perl -pi -e "s/install/#install/" /etc/exports

# Export /home and OpenHPC public packages from master server
[sms](*\#*) echo "/home *(rw,no_subtree_check,fsid=10,no_root_squash)" >> /etc/exports
[sms](*\#*) echo "/opt/ohpc/pub *(ro,no_subtree_check,fsid=11)" >> /etc/exports
[sms](*\#*) exportfs -a
[sms](*\#*) systemctl restart nfs-server
[sms](*\#*) systemctl enable nfs-server

# Enable NTP time service on computes and identify master host as local NTP server
[sms](*\#*) chroot $CHROOT systemctl enable ntpd
[sms](*\#*) echo "server ${sms_ip}" >> $CHROOT/etc/ntp.conf

\end{lstlisting}
% end_ohpc_run

