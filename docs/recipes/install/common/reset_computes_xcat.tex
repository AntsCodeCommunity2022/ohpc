Prior to booting the {\em compute} hosts, we configure them to use PXE as their
next boot mode. After the initial PXE, ensuing boots will return to using the default boot device
specified in the BIOS.

% begin_ohpc_run
% ohpc_comment_header Set nodes to netboot \ref{sec:boot_computes}
\begin{lstlisting}[language=bash,keywords={},upquote=true]
[sms](*\#*) rsetboot compute net
\end{lstlisting} 
% end_ohpc_run

At this point, the {\em master} server should be able to boot the newly defined
compute nodes. This is done by using the \texttt{rpower} \xCAT{} command
leveraging IPMI protocol set up during the the {\em compute} node definition
in \S~\ref{sec:xcat_add_nodes}. The following power cycles each of the
desired hosts.


% begin_ohpc_run
% ohpc_comment_header Boot compute nodes
\begin{lstlisting}[language=bash,keywords={},upquote=true]
[sms](*\#*) rpower compute reset
\end{lstlisting} 
% end_ohpc_run

Once kicked off, the boot process should take about \nottoggle{isxCATstateful}{5
minutes (depending on BIOS post times)}{5 to 10 minutes}.  You can monitor the
provisioning by using the \texttt{rcons} command, which displays serial console
for a selected node. Note that the escape sequence
is \texttt{CTRL-e c .} typed sequentially. 

Succesful provisioning can be verified by a parallel command on the compute
nodes. The default install provides two such tools: \xCAT{}-provided
\texttt{psh} command, which uses \xCAT{} node names and groups,
and \texttt{pdsh}, which works independently.  For example, to run a command on
the newly imaged compute hosts using \texttt{psh}, execute the following:

\begin{lstlisting}[language=bash]
[sms](*\#*) psh compute uptime
c1:  12:56:50 up 14 min,  0 users,  load average: 0.00, 0.01, 0.04
c2:  12:56:50 up 13 min,  0 users,  load average: 0.00, 0.02, 0.05
c3:  12:56:50 up 14 min,  0 users,  load average: 0.00, 0.02, 0.05
c4:  12:56:50 up 14 min,  0 users,  load average: 0.00, 0.01, 0.04
\end{lstlisting}
Note that the equivalent \texttt{pdsh} command is \texttt{pdsh -w c[1-4] uptime}. 
