\clustershell{} is an event-based Python library to execute commands in parallel
across cluster nodes. Installation and basic configuration defining three node
groups ({\em adm}, {\em compute}, and {\em all}) is as follows:

% begin_ohpc_run
% ohpc_validation_newline
% ohpc_command if [[ ${enable_clustershell} -eq 1 ]];then
% ohpc_indent 5
% ohpc_validation_comment Install clustershell
\begin{lstlisting}[language=bash,keywords={},upquote=true]
# Install ClusterShell
[sms](*\#*) (*\install*) clustershell-ohpc

# Setup node definitions
[sms](*\#*) cd /etc/clustershell/groups.d
[sms](*\#*) mv local.cfg local.cfg.orig
[sms](*\#*) echo "adm: ${sms_name}" > local.cfg
[sms](*\#*) echo "compute: ${compute_prefix}[1-${num_computes}]" >> local.cfg
[sms](*\#*) echo "all: @adm,@compute" >> local.cfg
\end{lstlisting}
% ohpc_indent 0
% ohpc_command fi
% end_ohpc_run

