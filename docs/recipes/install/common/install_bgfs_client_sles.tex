To add optional support for mounting \beegfs{} file systems, an 
additional external \pkgmgr{} repository must be configured. In this recipe, it is
assumed that the \beegfs{} file system is hosted by servers that are pre-existing
and are not part of the install process. Starting the client service triggers
a build of a kernel module, hence the kernel module development packages must be
installed first. As with \lustre{}, a default \baseOS{} environment may not allow 
loading of the necessary \beegfs{} kernel modules. Consequently, the example below 
includes steps which update the /etc/modprobe.d/10-unsupported-modules.conf file 
to allow loading of the necessary modules.

% begin_ohpc_run
% ohpc_validation_newline
% ohpc_command if [[ ${enable_bgfs_client} -eq 1 ]];then
% ohpc_indent 5
\begin{lstlisting}[language=bash,literate={-}{-}1,keywords={},upquote=true]
[sms](*\#*) cd /etc/zypp/repos.d
[sms](*\#*) wget ${beegfs_repo}
[sms](*\#*) (*\install*) kernel-devel
[sms](*\#*) (*\install*) beegfs-client beegfs-helperd beegfs-utils

[sms](*\#*) /opt/beegfs/sbin/beegfs-setup-client -m ${sysmgmtd_host}
[sms](*\#*) systemctl start beegfs-helperd

# Update configuration to allow BeeGFS modules to be loaded on master host
[sms]# perl -pi -e "s/^allow_unsupported_modules 0/allow_unsupported_modules 1/" \
/etc/modprobe.d/10-unsupported-modules.conf
[sms](*\#*) systemctl start beegfs-client
\end{lstlisting}
% ohpc_indent 0
% ohpc_command fi
% \end_ohpc_run
