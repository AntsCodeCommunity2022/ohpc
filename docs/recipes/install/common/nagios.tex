\Nagios{} is an open source infrastructure network monitoring package designed
to watch servers, switches, and various services and offers user-defined
alerting facilities for monitoring various aspects of an HPC
cluster. The core \Nagios{} daemon and a variety of monitoring plugins
are provided by the underlying OS distro and the following commands can
be used to install and configure a \Nagios{} server on the {\em
master} node, and add the facility to run tests and gather metrics
from provisioned {\em compute} nodes. This simple configuration
example is intended to be illustrative to walk through defining a
compute host group and enabling an ssh check for the computes. Users
are encouraged to
consult \Nagios{} \href{https://assets.nagios.com/downloads/nagioscore/docs/nagioscore/4/en/}{\color{blue}
documentation} for more information and can install additional plugins
as desired on login nodes, service nodes, or compute hosts.

% begin_ohpc_run
% ohpc_comment_header Configure Nagios on SMS and computes \ref{sec:add_nagios}
% ohpc_command if [[ ${enable_nagios} -eq 1 ]];then
% ohpc_indent 5
% ohpc_validation_comment Install Nagios on master and vnfs image
\begin{lstlisting}[language=bash,keywords={},upquote=true]
# Install nagios, nrep, and all available plugins on master host
[sms](*\#*) (*\install*) --skip-broken nagios nrpe nagios-plugins-*

# Install nrpe and an example plugin into compute node image
[sms](*\#*) (*\chrootinstall*) nrpe nagios-plugins-ssh

# Enable and configure Nagios NRPE daemon in compute image 
[sms](*\#*) chroot $CHROOT systemctl enable nrpe
[sms](*\#*) perl -pi -e "s/^allowed_hosts=/# allowed_hosts=/" $CHROOT/etc/nagios/nrpe.cfg
[sms](*\#*) echo "nrpe : ${sms_ip}  : ALLOW"    >> $CHROOT/etc/hosts.allow
[sms](*\#*) echo "nrpe : ALL : DENY"            >> $CHROOT/etc/hosts.allow

# Copy example Nagios config file to define a compute group and ssh check
# (note: edit as desired to add all desired compute hosts)
[sms](*\#*) cp /opt/ohpc/pub/examples/nagios/compute.cfg /etc/nagios/objects
# Register the config file with nagios
[sms](*\#*) echo "cfg_file=/etc/nagios/objects/compute.cfg" >> /etc/nagios/nagios.cfg

# Update location of mail binary for alert commands
[sms](*\#*) perl -pi -e "s/ \/bin\/mail/ \/usr\/bin\/mailx/g" /etc/nagios/objects/commands.cfg

# Update email address of contact for alerts
[sms](*\#*) perl -pi -e "s/nagios\@localhost/root\@${sms_name}/" /etc/nagios/objects/contacts.cfg

# Add check_ssh command for remote hosts
[sms](*\#*) echo command[check_ssh]=/usr/lib64/nagios/plugins/check_ssh localhost $CHROOT/etc/nagios/nrpe.cfg

# define password for nagiosadmin to be able to connect to web interface
[sms](*\#*) htpasswd -bc /etc/nagios/passwd nagiosadmin ${nagios_web_password}

# Enable Nagios on master, and configure
[sms](*\#*) systemctl enable nagios
[sms](*\#*) systemctl start nagios
# Update permissions on ping command to allow nagios user to execute
[sms](*\#*) chmod u+s `which ping`
\end{lstlisting}
% ohpc_indent 0
% ohpc_command fi
% end_ohpc_run

