\Nagios{} is an open source infrastructure monitoring package that monitors
servers, switches, applications, and services and offers user-defined alerting
facilities. As provided by \OHPC{}, it consists of a base monitoring daemon and
a set of plug-ins for monitoring various aspects of an HPC cluster.  The
following commands can be used to install and configure a \Nagios{} server on the {\em
master} node, and add the facility to run tests and gather metrics from
provisioned {\em compute} nodes.

% begin_ohpc_run
% ohpc_validation_newline
% ohpc_command if [ ${enable_nagios} -eq 1 ];then
% ohpc_indent 5
% ohpc_validation_comment Install Nagios on master and vnfs image
\begin{lstlisting}[language=bash,keywords={},upquote=true]
# Install Nagios base, Remote Plugin Engine, and Plugins on Master
[master](*\#*) (*\groupinstall*) ohpc-nagios

# Also install in compute node image
[master](*\#*) (*\groupchrootinstall*) ohpc-nagios

# Enable NRPE on compute image, and configure
[master](*\#*) chroot $CHROOT systemctl enable nrpe
[master](*\#*) chroot $CHROOT perl -pi -e "s/^allowed_hosts=/# allowed_hosts=/" /etc/nagios/nrpe.cfg
[master](*\#*) echo "nrpe 5666/tcp # NRPE" >> $CHROOT/etc/services
[master](*\#*) echo "nrpe : nagiosserver : ALLOW" >> $CHROOT/etc/hosts.allow
[master](*\#*) echo "nrpe : ALL : DENY" >> $CHROOT/etc/hosts.allow
[master](*\#*) echo "${master_ip}    nagiosserver" >> $CHROOT/etc/hosts
[master](*\#*) chroot $CHROOT /usr/sbin/useradd -c "NRPE user for the NRPE service" -d /var/run/nrpe \
        -r -g nrpe -s /sbin/nologin nrpe
[master](*\#*) chroot $CHROOT /usr/sbin/groupadd -r nrpe

# Enable Nagios on master, and configure
[master](*\#*) chkconfig nagios on
[master](*\#*) systemctl start nagios
[master](*\#*) chmod u+s `which ping`

# Configure remote services to test on compute nodes
[master](*\#*) cat <<SERVICES > /etc/nagios/conf.d/services.cfg
define service{
        use                     generic-service
        hostgroup_name          compute
        service_description     CPU Load
        check_command           check_nrpe!check_load
        }

define service{
        use                     generic-service
        hostgroup_name          compute
        service_description     Total Processes
        check_command           check_nrpe!check_total_procs
        }

define service{
        use                     generic-service
        hostgroup_name          compute
        service_description     Current Users
        check_command           check_nrpe!check_users
        }

define service{
        use                     generic-service
        hostgroup_name          compute
        service_description     SSH Monitoring
        check_command           check_nrpe!check_ssh
        }
SERVICES

# Define compute nodes as hosts to monitor
[master](*\#*) cat <<HOSTS > /etc/nagios/conf.d/hosts.cfg
## Linux Host Template ##
define host{
name                            linux-box     ; Name of this template
use                             generic-host  ; Inherit default values
check_period                    24x7
check_interval                  5
retry_interval                  1
max_check_attempts              10
check_command                   check-host-alive
notification_period             24x7
notification_interval           30
notification_options            d,r
contact_groups                  admins
register                        0             ; DONT REGISTER THIS - ITS A TEMPLATE
}

define hostgroup {
hostgroup_name                  compute
alias                           compute nodes
members                         ${c_name[0]},${c_name[1]},${c_name[2]},${c_name[3]}
}

define host{
use                             linux-box     ; Inherit default values from a template
host_name                       ${c_name[0]}  ; The name we're giving to this server
alias                           ${c_name[0]}  ; A longer name for the server
address                         ${c_ip[0]}    ; IP address of Remote Linux host
}
define host{
use                             linux-box     ; Inherit default values from a template
host_name                       ${c_name[1]}  ; The name we're giving to this server
alias                           ${c_name[1]}  ; A longer name for the server
address                         ${c_ip[1]}    ; IP address of Remote Linux host
}
define host{
use                             linux-box     ; Inherit default values from a template
host_name                       ${c_name[2]}  ; The name we're giving to this server
alias                           ${c_name[2]}  ; A longer name for the server
address                         ${c_ip[2]}    ; IP address of Remote Linux host
}
define host{
use                             linux-box     ; Inherit default values from a template
host_name                       ${c_name[3]}  ; The name we're giving to this server
alias                           ${c_name[3]}  ; A longer name for the server
address                         ${c_ip[3]}    ; IP address of Remote Linux host
}
HOSTS
\end{lstlisting}
% ohpc_indent 0
% ohpc_command fi
% end_ohpc_run

