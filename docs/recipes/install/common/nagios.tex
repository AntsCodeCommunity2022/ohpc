\Nagios{} is an open source infrastructure monitoring package that monitors
servers, switches, applications, and services and offers user-defined alerting
facilities. As provided by \OHPC{}, it consists of a base monitoring daemon and
a set of plug-ins for monitoring various aspects of an HPC cluster. The
following commands can be used to install and configure a \Nagios{} server on the {\em
master} node, and add the facility to run tests and gather metrics from
provisioned {\em compute} nodes.

% begin_ohpc_run
% ohpc_command if [[ ${enable_nagios} -eq 1 ]];then
% ohpc_indent 5
% ohpc_validation_comment Install Nagios on master and vnfs image
\begin{lstlisting}[language=bash,keywords={},upquote=true]
# Install Nagios meta-package on master host
[sms](*\#*) (*\install*) ohpc-nagios

# Install plugins into compute node image
[sms](*\#*) (*\chrootinstall*) nagios-plugins-all-ohpc nrpe-ohpc

# Enable and configure NRPE in compute image
[sms](*\#*) chroot $CHROOT systemctl enable nrpe
[sms](*\#*) perl -pi -e "s/^allowed_hosts=/# allowed_hosts=/" $CHROOT/etc/nagios/nrpe.cfg
[sms](*\#*) echo "nrpe 5666/tcp # NRPE"         >> $CHROOT/etc/services
[sms](*\#*) echo "nrpe : ${sms_ip}  : ALLOW"    >> $CHROOT/etc/hosts.allow
[sms](*\#*) echo "nrpe : ALL : DENY"            >> $CHROOT/etc/hosts.allow
[sms](*\#*) chroot $CHROOT /usr/sbin/useradd -c "NRPE user for the NRPE service" -d /var/run/nrpe \
        -r -g nrpe -s /sbin/nologin nrpe
[sms](*\#*) chroot $CHROOT /usr/sbin/groupadd -r nrpe

# Configure remote services to test on compute nodes
[sms](*\#*) mv /etc/nagios/conf.d/services.cfg.example /etc/nagios/conf.d/services.cfg

# Define compute nodes as hosts to monitor
[sms](*\#*) mv /etc/nagios/conf.d/hosts.cfg.example /etc/nagios/conf.d/hosts.cfg
[sms](*\#*) for ((i=0; i<$num_computes; i++)) ; do
              perl -pi -e "s/HOSTNAME$(($i+1))/${c_name[$i]}/ || s/HOST$(($i+1))_IP/${c_ip[$i]}/" \
              /etc/nagios/conf.d/hosts.cfg
           done

# Update location of mail binary for alert commands
[sms](*\#*) perl -pi -e "s/ \/bin\/mail/ \/usr\/bin\/mailx/g" /etc/nagios/objects/commands.cfg

# Update email address of contact for alerts
[sms](*\#*) perl -pi -e "s/nagios\@localhost/root\@${sms_name}/" /etc/nagios/objects/contacts.cfg

# Add check_ssh command for remote hosts
[sms](*\#*) echo command[check_ssh]=/usr/lib64/nagios/plugins/check_ssh localhost \
        >> $CHROOT/etc/nagios/nrpe.cfg

# define password for nagiosadmin to be able to connect to web interface
[sms](*\#*) htpasswd -bc passwd nagiosadmin ${nagios_web_password}

# Enable Nagios on master, and configure
[sms](*\#*) chkconfig nagios on
[sms](*\#*) systemctl start nagios
[sms](*\#*) chmod u+s `which ping`
\end{lstlisting}
% ohpc_indent 0
% ohpc_command fi
% end_ohpc_run

