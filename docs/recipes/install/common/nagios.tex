\Nagios{} is an open source infrastructure monitoring package that monitors
servers, switches, applications, and services and offers user-defined alerting
facilities. As provided by \OHPC{}, it consists of a base monitoring daemon and
a set of plug-ins for monitoring various aspects of an HPC cluster.  The
following commands can be used to install and configure a \Nagios{} server on the {\em
master} node, and add the facility to run tests and gather metrics from
provisioned {\em compute} nodes.

% begin_ohpc_run
% ohpc_validation_newline
% ohpc_command if [ ${enable_nagios} -eq 1 ];then
% ohpc_indent 5
% ohpc_validation_comment Install Nagios on master and vnfs image
\begin{lstlisting}[language=bash,keywords={},upquote=true]
# Install Nagios base, Remote Plugin Engine, and Plugins on Master
[master](*\#*) (*\groupinstall*) ohpc-nagios

# Also install in compute node image
[master](*\#*) (*\groupchrootinstall*) ohpc-nagios

# Enable NRPE on compute image, and configure
[master](*\#*) chroot $CHROOT systemctl enable nrpe
[master](*\#*) chroot $CHROOT perl -pi -e "s/^allowed_hosts=/# allowed_hosts=/" /etc/nagios/nrpe.cfg
[master](*\#*) echo "nrpe 5666/tcp # NRPE" >> $CHROOT/etc/services
[master](*\#*) echo "nrpe : nagiosserver : ALLOW" >> $CHROOT/etc/hosts.allow
[master](*\#*) echo "nrpe : ALL : DENY" >> $CHROOT/etc/hosts.allow
[master](*\#*) echo "${master_ip}    nagiosserver" >> $CHROOT/etc/hosts
[master](*\#*) chroot $CHROOT /usr/sbin/useradd -c "NRPE user for the NRPE service" -d /var/run/nrpe \
        -r -g nrpe -s /sbin/nologin nrpe
[master](*\#*) chroot $CHROOT /usr/sbin/groupadd -r nrpe

# Enable Nagios on master, and configure
[master](*\#*) chkconfig nagios on
[master](*\#*) systemctl start nagios
[master](*\#*) chmod u+s `which ping`

# Configure remote services to test on compute nodes
[master](*\#*) mv /etc/nagios/conf.d/services.cfg.example /etc/nagios/conf.d/services.cfg

# Define compute nodes as hosts to monitor
[master](*\#*) mv /etc/nagios/conf.d/hosts.cfg.example /etc/nagios/conf.d/hosts.cfg
[master](*\#*) for I in {0..3}; do
              perl -pi -e "s/HOSTNAME$(($I+1))/${c_name[$I]}/ || s/HOST$(($I+1))_IP/${c_ip[$I]}/" \
              /etc/nagios/conf.d/hosts.cfg
           done
\end{lstlisting}
% ohpc_indent 0
% ohpc_command fi
% end_ohpc_run

