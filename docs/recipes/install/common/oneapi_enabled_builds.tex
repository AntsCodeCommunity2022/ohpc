In addition to the 3rd party development libraries built using the open source
toolchains mentioned in \S\ref{sec:3rdparty}, \OHPC{} also provides {\em
  optional} compatible builds for use with the compilers and MPI stack included
in newer versions of the \IntelR{}~oneAPI HPC Toolkit (using the {\em classic}
compiler variants).  These
packages provide a similar hierarchical user
environment experience as other compiler and MPI families present in \OHPC{}.

To take advantage of the available builds, \OHPC{} provides a convenience
package to enable the oneAPI repository locally along with compatability
packages that integrate oneAPI-generated compiler and MPI modulefiles within
the standard \OHPC{} user environment. To enable and the \IntelR{} oneAPI
repository and install minimum compiler and MPI requirements for \OHPC{}
packaging, issue the following:

% begin_ohpc_run
% ohpc_comment_header Install Intel oneAPI tools \ref{sec:3rdparty_intel}
% ohpc_command if [[ ${enable_intel_packages} -eq 1 ]];then
% ohpc_indent 5
\begin{lstlisting}[language=bash,keywords={},upquote=true,keepspaces]
# Enable Intel oneAPI and install OpenHPC compatibility packages
[sms](*\#*) (*\install*) intel-oneapi-toolkit-release-ohpc
[sms](*\#*) (*\install*) intel-compilers-devel-ohpc
[sms](*\#*) (*\install*) intel-mpi-devel-ohpc
\end{lstlisting}


\begin{center}
\begin{tcolorbox}[]
As noted in \S\ref{sec:master_customization}, the default installation path for
\OHPC{} (\texttt{/opt/ohpc/pub}) is exported over NFS from the {\em master} to the 
compute nodes, but the \IntelR{} oneAPI HPC Toolkit packages install to a top-level path of 
\texttt{/opt/intel}. To make the \IntelR{} compilers available to the compute 
nodes one must add an additional NFS export
for \texttt{/opt/intel} that is mounted on desired compute nodes.
\end{tcolorbox}
\end{center}

\noindent To enable all 3rd party builds available in \OHPC{} that are compatible with
\IntelR{}~Parallel Studio, issue the following:


% ohpc_command if [[ ${enable_opa} -eq 1 ]];then
% ohpc_indent 10
\begin{lstlisting}[language=bash,keywords={},upquote=true,keepspaces]
# Optionally, choose the Omni-Path enabled build for MVAPICH2. Otherwise, skip to retain IB variant
[sms](*\#*) (*\install*) mvapich2-psm2-intel-ohpc
\end{lstlisting}
% ohpc_indent 5
% ohpc_command fi

\iftoggle{isSLES_ww_slurm_x86}{\clearpage}

\begin{lstlisting}[language=bash,keywords={},upquote=true,keepspaces]
# Install 3rd party libraries/tools meta-packages built with Intel toolchain
[sms](*\#*) (*\install*) ohpc-intel-serial-libs
[sms](*\#*) (*\install*) ohpc-intel-geopm
[sms](*\#*) (*\install*) ohpc-intel-io-libs
[sms](*\#*) (*\install*) ohpc-intel-perf-tools
[sms](*\#*) (*\install*) ohpc-intel-python3-libs
[sms](*\#*) (*\install*) ohpc-intel-mpich-parallel-libs
[sms](*\#*) (*\install*) ohpc-intel-mvapich2-parallel-libs
[sms](*\#*) (*\install*) ohpc-intel-openmpi4-parallel-libs
[sms](*\#*) (*\install*) ohpc-intel-impi-parallel-libs
\end{lstlisting}
% ohpc_indent 0
% ohpc_command fi
% end_ohpc_run

