% SLES-specific addition
Additionally, note that a default \baseOS{} environment may not allow loading of
the necessary \Lustre{} kernel modules. Consequently, the example below includes
steps which update the {\texttt /etc/modprobe.d/10-unsupported-modules.conf}
file to allow loading of the necessary modules.

% begin_ohpc_run
% ohpc_validation_newline
% ohpc_validation_comment Enable Optional packages
% ohpc_validation_newline
% ohpc_command if [[ ${enable_lustre_client} -eq 1 ]];then
% ohpc_indent 5

% ohpc_validation_comment Install Lustre client on master
\begin{lstlisting}[language=bash,keywords={},upquote=true]
# Add Lustre client software to master host
[sms](*\#*) (*\install*) lustre-client-ohpc

# Update configuration to allow Lustre modules to be loaded on master host
[sms](*\#*) perl -pi -e "s/^allow_unsupported_modules 0/allow_unsupported_modules 1/" \
     /etc/modprobe.d/10-unsupported-modules.conf
\end{lstlisting}
% end_ohpc_run


% begin_ohpc_run
% ohpc_validation_comment Enable lustre in WW compute image
\begin{lstlisting}[language=bash,keywords={},upquote=true]
# Include Lustre client software in compute image
[sms](*\#*) (*\chrootinstall*) lustre-client-ohpc

# Update configuration to allow Lustre modules to be loaded on compute hosts
[sms](*\#*) perl -pi -e "s/^allow_unsupported_modules 0/allow_unsupported_modules 1/" \
     $CHROOT/etc/modprobe.d/10-unsupported-modules.conf

# Include mount point and file system mount in compute image
[sms](*\#*) mkdir $CHROOT/mnt/lustre
[sms](*\#*) echo "${mgs_fs_name} /mnt/lustre lustre defaults,_netdev,localflock 0 0" >> $CHROOT/etc/fstab
\end{lstlisting}
% end_ohpc_run
