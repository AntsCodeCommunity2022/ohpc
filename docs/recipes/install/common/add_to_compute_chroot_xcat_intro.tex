% -*- mode: latex; fill-column: 120; -*- 

The \texttt{genimage} process used in the previous step is designed to provide a minimal \baseOS{} configuration. Next,
we add additional components to include resource management client services, \InfiniBand{} drivers, and other additional
packages to support the default \OHPC{} environment. This process augments the chroot-based install performed by
\texttt{genimage} to modify the base provisioning image and requires access the BOS and \OHPC{} repositories to resolve
package install requests. The following steps are used to first enable the necessary package repositories for use within
the chroot.

% begin_ohpc_run
% ohpc_comment_header Enable repos in chroot \ref{sec:add_components}
\begin{lstlisting}[language=bash,literate={-}{-}1,keywords={},upquote=true]
[sms](*\#*) yum --installroot=$CHROOT clean expire-cache
[sms](*\#*) cp /etc/yum.repos.d/OpenHPC.repo $CHROOT/etc/yum.repos.d
[sms](*\#*) cp /etc/yum.repos.d/epel.repo $CHROOT/etc/yum.repos.d
\end{lstlisting}
% end_ohpc_run

\noindent Next, install a base compute package and disable firewall:
% begin_ohpc_run
% ohpc_comment_header Add OpenHPC base components to compute image \ref{sec:add_components}
\begin{lstlisting}[language=bash,literate={-}{-}1,keywords={},upquote=true]
# Install compute node base meta-package
[sms](*\#*) (*\chrootinstall*) ohpc-base-compute
# Disable firewall for computes
[sms](*\#*) chroot $CHROOT systemctl disable firewalld
\end{lstlisting}
% end_ohpc_run

\noindent Now, we can include additional components to the compute instance using
\texttt{\pkgmgr{}} to install into the chroot location defined previously:
